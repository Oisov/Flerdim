\documentclass[a4paper, norsk, 12pt]{article}

\usepackage{xparse}
% Defines a simple bold text, \oppgave[text] produces Oppgave 1 (text). 
\NewDocumentCommand{\oppgave}{o}{%
  % <code>%
  \noindent\\
  % \textbf{\large Oppgave~\stepcounter{problemMA1103}\arabic{problemMA1103}}
  \textbf{Oppgave~\stepcounter{problem\arabic{section}}\arabic{problem\arabic{section}}}
  \IfNoValueTF{#1}
  {}
  {(#1)}%
  \medskip\\\noindent
  % <code>
}
\usepackage{totcount}
\regtotcounter{section}

\usepackage{nameref}
\newcommand{\seksjon}[1]{\section{#1}\label{sec:\arabic{section}}\newcounter{problem\arabic{section}}}

\usepackage{tikz}
\usepackage{pgfplots}
 
\begin{document}

\seksjon{Integrals}

\oppgave

\seksjon{Derivatives}

\oppgave
\oppgave
\oppgave
\seksjon{Green's theorem}

\oppgave
\oppgave
\seksjon{Gauss' Theorem}

\oppgave
\oppgave
\seksjon{Stokes Theorem}

\oppgave
\oppgave
\oppgave
\oppgave


The total number of chapters is \total{section}, in \nameref{sec:1} there are
\arabic{problem1} problems and in \nameref{sec:5} there are
\arabic{problem5} problems.

\begin{tikzpicture}
  \foreach \i in {1,...,\arabic{section}}
  {
    \node at (0, -\i) {\nameref{sec:\i}: \arabic{problem\i}};
  }
\end{tikzpicture} 

\begin{figure}[htbp!]
  \centering
\begin{tikzpicture}
    \begin{axis}[
    xbar,
    y=-0.5cm,
    bar width=0.3cm,
    y axis line style = { opacity = 0 },
    axis x line       = none,
    tickwidth         = 0pt,
    enlarge y limits  = 0.2,
    enlarge x limits  = 0.02,
    ytick = data,
    nodes near coords,
    symbolic y coords = {%
      Integrals, 
      Derivatives,
      Green's theorem,
      Gauss' theorem, 
      Stokes theorem}
    ]
    \addplot[fill=black] coordinates {%
      (\arabic{problem1},Integrals)
      (\arabic{problem2},Derivatives)
      (\arabic{problem3},Green's theorem)  
      (\arabic{problem4},Gauss' theorem)  
      (\arabic{problem5},Stokes theorem)};
%    \legend{Topics, Posts}
    \end{axis}
    \end{tikzpicture}
  \end{figure}

\end{document}