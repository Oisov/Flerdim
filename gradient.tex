\seksjon{Gradient}

\oppgave[V2015, Oppgave 2]

Funksjonen $f \colon \R^2 \to \R$ er deriverbar. Vi vet også at den
retningsderiverte i $(1, 0)$ langs positiv $x$-akse er $5$, og at den
retningsderiverte i $(1, 0)$ langs linja $y = x - 1$ i retning av positiv $y$,
er $- 2$. Hva er gradienten til $f$ i $(1, 0)$?

\oppgave[K2014, Oppgave 5]

Anta at et fjell har form som en elliptisk paraboloide $z = c - ax^2 - by^2$,
der $a$, $b$ og $c$ er positive konstanter, $x$ og $y$ er øst–vest og nord–sør
koordinater på kartet, mens $z$ er høyden over havet.
%
\begin{enumerate}
  \item I hvilken retning stiger høyden mest i punktet $(1, 1)$?
  \item Hvis en klinkekule slippes i $(1, 1, c - a - b)$, i hvilken retning vil den begynne
    å trille?
\end{enumerate}

\oppgave[V2014, Oppgave 2]

La $T(x,y,z) = e^{x + 2y + 3z}$ være temperaturen i et romlig område om origo.
\medskip

I hvilke retninger fra $(0,0,0)$ vokser og avtar temperaturen mest? \medskip

Hva er den retningsderiverte i disse retningene?

\oppgave[V2012, Oppgave 2] Finn den retningsderiverte av funksjonen $f(x,y,z) =
e^{-x^2}y - \log(1 + e^z)$ i punktet $(1,1,0)$ i retningen fra $(1,1,0)$ til $(-1,2,1)$.