\documentclass[screen, aspectratio=43, serif]{beamer}

%\includeonlyframes{current}

\usepackage[T1]{fontenc}
\usepackage[utf8]{inputenc}

% Use the NTNU-temaet for beamer 
% \usetheme[style=ntnu|simple|vertical|horizontal, language=bm|nn|en]{ntnu2015}

\usetheme[language=norsk]{Boadilla}
\setbeamertemplate{frametitle}{\centerline{\bfseries\insertframetitle}}
% \setbeamertemplate{frametitle}[default][center]

\newenvironment<>{intuisjon}{%
  \setbeamercolor{block title}{fg=black,bg=yellow!75!black}%
  \begin{block}{Intuisjon}}{\end{block}}

\newenvironment<>{oppgave}[1]{%
  \setbeamercolor{block title}{fg=black,bg=green!60!black}%
  \begin{block}#2{#1}}{\end{block}}


% ===================================
% Mathematics
% ===================================
    \usepackage{amsmath, amssymb, amsthm, mathtools}   % Mijøer og verktøy for å
    % skrive matematikk
    \usepackage{esdiff, esint}
    
\let\div\relax
\DeclareMathOperator{\div}{div}
\DeclareMathOperator{\curl}{curl}
    

    \DeclareMathOperator{\2d-curl}{2d-curl}
    \DeclareMathOperator{\grad}{grad}

    \newcommand{\vek}[1]{\mathbf{#1}}
    \newcommand{\unitvek}[1]{\vek{\hat{#1}}}
    \newcommand{\normal}{\mathbf{n}}

    \usepackage[scr]{rsfso}
    \newcommand{\La}{\mathscr{L}}
    
    \DeclarePairedDelimiter\abs{\lvert}{\rvert}%
    \DeclarePairedDelimiter\norm{\lVert}{\rVert}%

    \makeatletter
    \newcommand*{\rom}[1]{\expandafter\@slowromancap\romannumeral #1@}
    \newcommand*{\dd}{\mathop{}\!{\operator@font d}}
    \makeatother
    
    \newcommand{\F}{\vek{F}}
    \newcommand{\rr}{\vek{r}}
    \newcommand{\R}{\mathbb{R}}
    \newcommand{\N}{\mathbb{N}}
    
    \newcommand{\I}{\vek{i}}
    \newcommand{\J}{\vek{j}}
    \newcommand{\K}{\vek{k}}
    \newcommand{\X}{\vek{x}}
    \newcommand{\Y}{\vek{y}}
    \newcommand{\A}{\vek{a}}
    \newcommand{\B}{\vek{b}}
    
    \newcommand{\dt}{\mathop{}\! \mathrm{d} t\mathop{}\! }
    \newcommand{\dT}{\mathop{}\! \mathrm{d} \theta\mathop{}\! }
    \newcommand{\du}{\mathop{}\! \mathrm{d} u\mathop{}\! }
    \newcommand{\dv}{\mathop{}\! \mathrm{d} v\mathop{}\! }
    \newcommand{\ds}{\mathop{}\! \mathrm{d} s\mathop{}\! }
    \newcommand{\dx}{\mathop{}\! \mathrm{d} x\mathop{}\! }
    \newcommand{\dy}{\mathop{}\! \mathrm{d} y\mathop{}\! }
    \newcommand{\dz}{\mathop{}\! \mathrm{d} z\mathop{}\! }
    \newcommand{\dr}{\mathop{}\! \mathrm{d} \vek{r}\mathop{}\! }
    \newcommand{\dA}{\mathop{}\! \mathrm{d} A\mathop{}\! }
    \newcommand{\dV}{\mathop{}\! \mathrm{d} V\mathop{}\! }
    \newcommand{\dS}{\mathop{}\! \mathrm{d} S\mathop{}\! }
    \newcommand{\dSS}{\mathop{}\! \mathrm{d} \vek{S}\mathop{}\! }


\usepackage{graphicx}
\usepackage{lmodern}% http://ctan.org/pkg/lm
% \usepackage{enumitem}

\usepackage{refcount}

\newcounter{saveenumi}
\newcommand{\saveenumerate}{%
  \stepcounter{saveenumi}%
  \label{saveenumi-\thesaveenumi}}
\newcommand{\restoreenumerate}{%
  \setcounterref{enumi}{saveenumi-\thesaveenumi}}


\usepackage{nameref}
\newcommand{\topicref}[1]{\hyperlink{subsec:#1}{\nameref{subsec:#1}}}

\usepackage{xcolor}

\setbeamerfont{caption}{size=\footnotesize}
\setbeamertemplate{caption}{\raggedright\insertcaption\par\centering}

 \usepackage{pgfplots,tikz}

\usetikzlibrary{shapes.geometric, arrows}

\tikzstyle{startstop} = [rectangle, rounded corners, minimum width=3cm, minimum
height=1cm,text centered, draw=black, fill=red!30]
\tikzstyle{starting} = [trapezium, trapezium left angle=70, trapezium right
angle=110, rounded corners, minimum width=1cm, minimum
height=1cm,text centered, draw=black, fill=blue!40]
\tikzstyle{io} = [trapezium,
trapezium left angle=70, trapezium right angle=110, minimum width=3cm, minimum
height=1cm, text centered, draw=black, fill=blue!30]
\tikzstyle{process} =
[rectangle, minimum width=3cm, minimum height=1cm, text centered, draw=black,
fill=orange!30]
\tikzstyle{decision} = [diamond, minimum width=3cm, minimum
height=1cm, text centered, draw=black, fill=green!30]

\tikzstyle{arrow} = [thick,->,>=stealth]

\newcommand*\circled[1]{\tikz[baseline=(char.base)]{
            \node[shape=circle,draw,inner sep=2pt] (char) {$#1$};}}

\title[s.ntnu.no/MA1103]{Flerdimhistorien på 90 minutter}
\subtitle{s.ntnu.no/MA1103}
\author[Ø. Søvik]{Øistein Søvik}

\date{Mai 2018}
%\date{} % To have an empty date

\begin{document}

\begin{frame}
    \titlepage
\end{frame}

\begin{frame}\centerline{
  \begin{tikzpicture}[scale=0.75]
    \begin{axis}[ xbar=0pt, /pgf/bar shift=0pt, legend style={ legend columns=4,
        at={(xticklabel cs:0.5)}, anchor=north, draw=none }, ytick={0,...,15},
      ytick style={draw=none},% <- added
      axis y line*=none, axis x line*=bottom, tick label
      style={font=\footnotesize}, legend style={font=\footnotesize}, label
      style={font=\footnotesize}, xtick style={draw=none},% <- added
      xtick={0,1,...,12}, width=.9\textwidth, bar width=3mm, y dir = reverse,
      xmin=0, xmax=13, area legend,
      y=5mm, enlarge y limits={abs=0.625},
      style={text=black}, every axis plot/.append style={fill},
      nodes near coords, nodes near coords,
      yticklabels={%
        {\topicref{Grenseverdier}},
        {\topicref{Gradient}},
        {\topicref{Epsilon-delta}},
        {\topicref{Kjerneregelen}},
        {\topicref{Taylor-approksimasjon}},
        {\topicref{Tangenter}},
        {\topicref{Kritiske-punkter}},
        {\topicref{Optimering}},
        {\topicref{Linjeintegral}},
        {\topicref{Konservative-vektorfelt}},
        {\topicref{Dobbelintegraler}},
        {\topicref{Integrasjonsrekkefolge}},
        {\topicref{Trippelintegral}},
        {\topicref{Greens-teorem}},
        {\topicref{Divergensteoremet}},
        {\topicref{Stokes-teorem}}}]
      \addplot[fill=red] coordinates {(8,0)};
      \addplot[fill=black] coordinates {(4,1)};
      \addplot[fill=black] coordinates {(1,2)};
      \addplot[fill=black] coordinates {(7,3)};
      \addplot[fill=black] coordinates {(1,4)};
      \addplot[fill=black] coordinates {(7,5)};
      \addplot[fill=black] coordinates {(11,6)};
      \addplot[fill=black] coordinates {(4,7)};
      \addplot[fill=black] coordinates {(3,8)};
      \addplot[fill=black] coordinates {(8,9)};
      \addplot[fill=black] coordinates {(7,10)};
      \addplot[fill=black] coordinates {(3,11)};
      \addplot[fill=black] coordinates {(4,12)};
      \addplot[fill=black] coordinates {(4,13)};
      \addplot[fill=black] coordinates {(8,14)};
      \addplot[fill=black] coordinates {(6,15)};
    \end{axis}
  \end{tikzpicture}}
  
  \textbf{Plan:}
  \begin{enumerate}
    \item Før pausen: Derivasjon 7 temaer
    \item Etter pausen: Integrasjon 7 temaer
    \item Hvert tema har inneholder introduksjon/intuisjon, oppgaveløsning.
  \end{enumerate}
\end{frame}

% % Special title page command to get a different background
\begin{frame}
    \titlepage
\end{frame}

\section{Oppgaver}

\begin{frame}{Innholdsfortegnelse}

\begin{itemize}
    \item Introduksjon
    \item Nyttige tips før eksamen
    \item Nyttige tips under eksamen
    \item Oppgaveregning
\end{itemize}Q
    
\end{frame}

\begin{frame}
    \frametitle{Før eksamen}
    \begin{itemize}
        \item For forståelsen: \url{http://mathinsight.org/thread/multivar}
        \item Matte 2: Oppgaveløsning på video. \url{https://video.adm.ntnu.no/serier/52f0b7e0294b4}
        \item Lag en frekvenstabell over tidligere gitte eksamensoppgaver
        \item Øv til eksamen mest mulig likt eksamen. Ta tiden, bruk LF sparsomt. 
        \item Matte 2 eksamener er $99\%$ likt Ma1103, regn de og!
    \end{itemize}
\end{frame}

\begin{frame}
  \frametitle{Før eksamen}
  \begin{enumerate}
    \item \textbf{Alltid} ny deloppgave på ny side
    \item Tegn store figurer 
    \item Stor frokost + rask mat under eksamen
    \item Slapp av! Selv om eksamen ikke går helt som planlagt har du fortsatt Delta. 
  \end{enumerate}
\end{frame}


\begin{frame}
  \begin{tikzpicture}
    \begin{axis}[ xbar=0pt, /pgf/bar shift=0pt, legend style={ legend columns=4,
        at={(xticklabel cs:0.5)}, anchor=north, draw=none }, ytick={0,...,15},
      ytick style={draw=none},% <- added
      axis y line*=none, axis x line*=bottom, tick label
      style={font=\footnotesize}, legend style={font=\footnotesize}, label
      style={font=\footnotesize}, xtick style={draw=none},% <- added
      xtick={0,1,...,12}, width=.9\textwidth, bar width=3mm, y dir = reverse,
      yticklabels={%
        {Grenseverdier}, {Gradient}, {Epsilon-Delta}, {Kjerneregelen},
        {Taylor-approksimasjon}, {Tangenter}, {Kritiske punkter}, {Optimering},
        {Linjeintegral}, {Konservative vektorfelt}, {Dobbelintegraler},
        {Integrasjonsrekkefølge}, {Trippelintegral}, {Greens teorem},
        {Divergensteoremet}, {Stokes teorem}}, xmin=0, xmax=13, area legend,
      y=5mm, enlarge y limits={abs=0.625}, nodes near coords, nodes near coords
      style={text=black}, every axis plot/.append style={fill} ]
      \addplot[fill=black] coordinates {(8,0)}; \addplot[fill=black] coordinates
      {(4,1)}; \addplot[fill=red] coordinates {(1,2)}; \addplot[fill=black]
      coordinates {(7,3)}; \addplot[fill=black] coordinates {(1,4)};
      \addplot[fill=black] coordinates {(7,5)}; \addplot[fill=black] coordinates
      {(11,6)}; \addplot[fill=black] coordinates {(4,7)}; \addplot[fill=black]
      coordinates {(3,8)}; \addplot[fill=black] coordinates {(8,9)};
      \addplot[fill=black] coordinates {(7,10)}; \addplot[fill=black]
      coordinates {(3,11)}; \addplot[fill=black] coordinates {(4,12)};
      \addplot[fill=black] coordinates {(4,13)}; \addplot[fill=black]
      coordinates {(8,14)}; \addplot[fill=black] coordinates {(6,15)};
    \end{axis}
  \end{tikzpicture}
\end{frame}

%%% Local Variables:
%%% mode: latex
%%% TeX-master: "main"
%%% End:


% 14/20 Done

% Derivasjon
{%
  \setbeamercolor{background canvas}{bg=black}
  \setbeamercolor{frametitle}{fg=white}
  \begin{frame}
  \frametitle{\bfseries Derivasjon}
  \section{Derivasjon}
  \begin{center}
  \begin{tikzpicture}
      \begin{axis} [
        xtick = {-50,-60},
        ytick = {-5,-6},
        ztick = {-3,-4},
        ticklabel style = {font = \scriptsize},
        axis line style={draw=none},
        tick style={draw=none}
      ]
      \addplot3 [surf, domain=0:360, samples=30] 
        { sin(x)*sin(y) };
      \end{axis}
    \end{tikzpicture}
    \visible<2>{\includegraphics[width=0.4\textwidth]{../img/pringles.png}}
   \end{center}
  \end{frame}%
}

%%% Local Variables:
%%% mode: latex
%%% TeX-master: "main"
%%% End:
              % Done
% \begin{frame}
  \begin{tikzpicture}
    \begin{axis}[ xbar=0pt, /pgf/bar shift=0pt, legend style={ legend columns=4,
        at={(xticklabel cs:0.5)}, anchor=north, draw=none }, ytick={0,...,15},
      ytick style={draw=none},% <- added
      axis y line*=none, axis x line*=bottom, tick label
      style={font=\footnotesize}, legend style={font=\footnotesize}, label
      style={font=\footnotesize}, xtick style={draw=none},% <- added
      xtick={0,1,...,12}, width=.9\textwidth, bar width=3mm, y dir = reverse,
      xmin=0, xmax=13, area legend,
      y=5mm, enlarge y limits={abs=0.625},
      style={text=black}, every axis plot/.append style={fill},
      nodes near coords, nodes near coords,
      yticklabels={%
        {\topicref{Grenseverdier}},
        {\topicref{Gradient}},
        {\topicref{Epsilon-delta}},
        {\topicref{Kjerneregelen}},
        {\topicref{Taylor-approksimasjon}},
        {\topicref{Tangenter}},
        {\topicref{Kritiske-punkter}},
        {\topicref{Optimering}},
        {\topicref{Linjeintegral}},
        {\topicref{Konservative-vektorfelt}},
        {\topicref{Dobbelintegraler}},
        {\topicref{Integrasjonsrekkefolge}},
        {\topicref{Trippelintegral}},
        {\topicref{Greens-teorem}},
        {\topicref{Divergensteoremet}},
        {\topicref{Stokes-teorem}}}]
      \addplot[fill=red] coordinates {(8,0)};
      \addplot[fill=black] coordinates {(4,1)};
      \addplot[fill=black] coordinates {(1,2)};
      \addplot[fill=black] coordinates {(7,3)};
      \addplot[fill=black] coordinates {(1,4)};
      \addplot[fill=black] coordinates {(7,5)};
      \addplot[fill=black] coordinates {(11,6)};
      \addplot[fill=black] coordinates {(4,7)};
      \addplot[fill=black] coordinates {(3,8)};
      \addplot[fill=black] coordinates {(8,9)};
      \addplot[fill=black] coordinates {(7,10)};
      \addplot[fill=black] coordinates {(3,11)};
      \addplot[fill=black] coordinates {(4,12)};
      \addplot[fill=black] coordinates {(4,13)};
      \addplot[fill=black] coordinates {(8,14)};
      \addplot[fill=black] coordinates {(6,15)};
    \end{axis}
  \end{tikzpicture}
\end{frame}

\begin{frame}
  \frametitle{Grenseverdier}
  \section{Grenseverdier}
  \label{subsec:Grenseverdier}
  En funksjon $f(x,y)$ er kontinuerlig i $(0,0)$ dersom
  %
  \begin{equation*}
    \lim_{(x,y)\to(0,0)} f(x,y) = f(0,0)\,,
  \end{equation*}
  %
  \emph{uavhengig} av hvilken retning du nærmer deg origo. \visible<2->{Anta du ønsker å undersøke
  om en funksjon på formen
  %
  \begin{equation*}
    f(x,y) :=
    \begin{cases}
      \frac{g(x,y)}{h(x,y)}, & \text{når} \quad (x,y) \neq (0,0), \\
      0, & \text{når} \quad (x,y) = (0,0)
    \end{cases},
  \end{equation*}
  er kontinuerlig.
  \begin{itemize}
    \item Forenkle $f(x,ax^b)$ mest mulig. Eksisterer $a$ og $b$
      slik at $\lim_{x \to 0}f(x,ax^b) \neq f(0,0)$? Da er funksjonen
      \emph{ikke} kontinuerlig.
    \item Inneholder $h(x,y)$ uttrykket $x^2 + y^2$? Forenkle $f(r \cos \theta,
      r \sin \theta)$. Undersøk om $\lim_{r \to 0} f(r \cos \theta, r\sin
      \theta) = 0$ for \emph{alle} vinkler, altså $\theta$.
  \end{itemize}}
\end{frame}

\begin{frame}
  \begin{oppgave}{V2016, Oppgave 1}
    %
    \begin{equation*}
      f(x,y) :=
      \begin{cases}
        \cfrac{yx^3}{3y^2 + x^6}, & \text{når} \quad (x,y) \neq (0,0), \\
        0, & \text{når} \quad (x,y) = (0,0)
      \end{cases}.
    \end{equation*}
    %
    Forklar hvorfor $f$ ikke er deriverbar i $(0,0)$, men de partiellderiverte
    $\diffp{f}{x}(0,0)$ og $\diffp{f}{y}(0,0)$ eksisterer.
  \end{oppgave}
    \only<1-4>{\textbf{Steg 1:} Forenkle $f(x,ax^b)$ mest mulig.}
   \only<5-7>{\textbf{Steg 2:} Eksisterer det nå en $a$ og $b$ slik at $\displaystyle\lim_{x \to 0} f(x, ax^b)
     \neq f(0,0) = 0$?}
   %
   \only<1-7>{
  \begin{equation*}
    \only<6->{\lim_{x\to 0}}\only<1-5>{f(x, ax^b)}\only<6->{f(x,ax^3)}
    \only<1>{= \frac{ax^bx^3}{3(ax^b)^2+x^6}}
    \only<2>{= \frac{ax^{3+b}}{3a^2x^{2b} + x^6}}
    \only<3>{= \frac{a}{3a^2x^{2b-(3+b)} + x^{6 - (3+b)}}}
    \only<4-5>{= \frac{a}{3a^2x^{b-3} + x^{3 - b}}}
    \only<6>{= \lim_{x \to 0} \frac{a}{3a^2x^{3-3}+x^{3-3}}}
    \only<7>{= \frac{a}{3a^2 + 1}}
  \end{equation*}}
  %
  \only<7>{Siden $f$ ikke er kontinuerlig kan den naturlig nok heller ikke være
    deriverbar!}
  \only<8>{%
    \begin{itemize}
        \item Notasjonen $\diffp{f}{x}(0,0)$ betyr \emph{ikke} den deriverte med
    hensyn på $x$ i origo!

    \item Den partiellderiverte er den \emph{retningsderiverte} av $f$ i retning
        $(1,0)$ altså langs $x$-aksen. Tilsvarende for $\partial f/\partial y(0,0)$.

   \item At funksjonen ikke er deriverbar betyr at det eksisterer en retning mot origo
 hvor den retningsderiverte ikke har et klart definert stigningstall. De
        retningsderiverte i $x$ og $y$ retningen kan såklart likevell eksistere.
    \end{itemize}}
  \only<9-12>{%
    Den \emph{retningsderiverte} til $f$ i punktet $\A=(0,0)$ og
    \emph{retning} $\rr$ er gitt ved
    \begin{align*}
      f'(\A; \rr) & = \lim_{h \to 0} \frac{f(\A + h \rr) - f(\A)}{h}\\
      \diffp{f}{x}(0,0) & =
      \only<9>{\lim_{h \to 0} \frac{f((0,0) + h (1,0)) - f(0,0)}{h}}
      \only<10>{\lim_{h \to 0} \frac{f(h,0) - f(0,0)}{h}} 
      \only<11>{\lim_{h \to 0} \frac{\frac{0\cdot h^3}{3\cdot 0^2 + h^6} - 0}{h}} 
      \only<12>{\lim_{h \to 0} \frac{0 - 0}{h} = 0} \\
      \diffp{f}{y}(0,0) & =
      \only<9>{\lim_{h \to 0} \frac{f((0,0) + h (0,1)) - f(0,0)}{h}}
      \only<10>{\lim_{h \to 0} \frac{f(0,h) - f(0,0)}{h}}
      \only<11>{\lim_{h \to 0} \frac{\frac{h \cdot 0^3}{3h^2 + 0^6} - 0}{h}} 
      \only<12>{\lim_{h \to 0} \frac{0 - 0}{h} = 0} 
    \end{align*}
  }
\end{frame}

\begin{frame}
  \begin{oppgave}{K2013, Oppgave 3}
    Vis at en av grensene under eksisterer mens den andre ikke gjør det
    %
    \begin{equation*}
      \lim_{(x,y)\to(0,0)} \frac{xy}{\sqrt{x^2+y^2}}
      \qquad
      \lim_{(x,y)\to(0,0)} \frac{xy}{x^2+y^2}
    \end{equation*}
  \end{oppgave}
  \only<1-4>{\textbf{Steg 1}: Uttrykket inneholder $x^2 + y^2$ så vi forenkler $f(r \cos
    \theta, r \sin \theta)$.}
  \only<5-6>{
  \textbf{Steg 2}: Undersøker om $\lim_{r \to 0} f(r \cos \theta, r\sin \theta)$ er uavhengig av $\theta$.} 
  \begin{align*}
  \only<1>{
    & \ \frac{(r\cos \theta) (r\sin \theta)}{\sqrt{(r\cos \theta)^2 + (r\sin \theta)^2}\,}
    && \ \frac{(r \cos \theta)(r\sin \theta)}{(r \cos \theta)^2 + (r \sin \theta)^2}}
       \only<2>{
     & \ \frac{r^2(\cos \theta) (\sin \theta)}{\sqrt{r^2(\cos^2\theta + \sin^2\theta)}\,}
    &&  \ \frac{r^2(\cos \theta)(\sin \theta)}{r^2(\cos^2\theta + \sin^2\theta)}}
       \only<3>{
     & \ \frac{r^2}{\sqrt{r^2}}(\cos \theta)(\sin \theta)
    && \ \frac{r^2}{r^2}(\cos \theta)(\sin \theta)}
       \only<4>{
    & \ r (\cos \theta)(\sin \theta)
    && \ (\cos \theta)(\sin \theta)}
       \only<5>{
    \lim_{r \to 0} & \ r (\cos \theta)(\sin \theta)
    && \lim_{r \to 0} \ (\cos \theta)(\sin \theta)}
              \only<6>{
    \underbrace{\lim_{r \to 0} \ r (\cos \theta)(\sin \theta)}_{\text{Ja, $0$ for alle $\theta$}}
    && \underbrace{\lim_{r \to 0} \ (\cos \theta)(\sin \theta)}_{\text{Nei, $\theta=0 \to 0$, $\theta=\pi/4 \to 1/2$.}}}
  \end{align*}
\end{frame}


%%% Local Variables:
%%% mode: latex
%%% TeX-master: "main"
%%% End:
           % Done 
% \begin{frame}
  \begin{tikzpicture}
    \begin{axis}[ xbar=0pt, /pgf/bar shift=0pt, legend style={ legend columns=4,
        at={(xticklabel cs:0.5)}, anchor=north, draw=none }, ytick={0,...,15},
      ytick style={draw=none},% <- added
      axis y line*=none, axis x line*=bottom, tick label
      style={font=\footnotesize}, legend style={font=\footnotesize}, label
      style={font=\footnotesize}, xtick style={draw=none},% <- added
      xtick={0,1,...,12}, width=.9\textwidth, bar width=3mm, y dir = reverse,
      xmin=0, xmax=13, area legend,
      y=5mm, enlarge y limits={abs=0.625},
      style={text=black}, every axis plot/.append style={fill},
      nodes near coords, nodes near coords,
      yticklabels={%
        {\topicref{Grenseverdier}},
        {\topicref{Gradient}},
        {\topicref{Epsilon-delta}},
        {\topicref{Kjerneregelen}},
        {\topicref{Taylor-approksimasjon}},
        {\topicref{Tangenter}},
        {\topicref{Kritiske-punkter}},
        {\topicref{Optimering}},
        {\topicref{Linjeintegral}},
        {\topicref{Konservative-vektorfelt}},
        {\topicref{Dobbelintegraler}},
        {\topicref{Integrasjonsrekkefolge}},
        {\topicref{Trippelintegral}},
        {\topicref{Greens-teorem}},
        {\topicref{Divergensteoremet}},
        {\topicref{Stokes-teorem}}}]
      \addplot[fill=gray] coordinates {(8,0)};
      \addplot[fill=dgreen] coordinates {(4,1)};
      \addplot[fill=black] coordinates {(1,2)};
      \addplot[fill=black] coordinates {(7,3)};
      \addplot[fill=black] coordinates {(1,4)};
      \addplot[fill=black] coordinates {(7,5)};
      \addplot[fill=black] coordinates {(11,6)};
      \addplot[fill=black] coordinates {(4,7)};
      \addplot[fill=black] coordinates {(3,8)};
      \addplot[fill=black] coordinates {(8,9)};
      \addplot[fill=black] coordinates {(7,10)};
      \addplot[fill=black] coordinates {(3,11)};
      \addplot[fill=black] coordinates {(4,12)};
      \addplot[fill=black] coordinates {(4,13)};
      \addplot[fill=black] coordinates {(8,14)};
      \addplot[fill=black] coordinates {(6,15)};
    \end{axis}
  \end{tikzpicture}
\end{frame}

\begin{frame}
  \subsection{Gradient}\label{subsec:Gradient}
  \frametitle{Gradient}
  Gradienten til et skalarfelt er et vektorfelt som i hvert
  punkt peker i retningen til den største økningen i skalarfeltet.
  %
  \begin{align*}
       \nabla f (x,y)
    &= \Bigl( \diffp{f}{x}, \ \diffp{f}{y}\Bigl) \\
       \nabla f (\rho, \theta, z)
    &= \Bigl( \diffp{f}{\rho},\
             \frac{1}{\rho}\diffp{f}{\varphi},
             \diffp{f}{z}
       \Bigl) \\
       \nabla f (r, \theta, \phi)
    &= \Bigl( \diffp{f}{r},\
             \frac{1}{r}\diffp{f}{\theta},\
             \frac{1}{r \sin\theta}\diff{f}{\varphi}
       \Bigl)
  \end{align*}
\end{frame}

\begin{frame}
  \begin{oppgave}{V2016, Oppgave 1}
Anta at et fjell har form som en elliptisk paraboloide $z = c - ax^2 - by^2$,
der $a$, $b$ og $c$ er positive konstanter, $x$ og $y$ er øst–vest og nord–sør
koordinater på kartet, mens $z$ er høyden over havet.
%
\begin{enumerate}
  \item I hvilken retning stiger høyden mest i punktet $(1, 1)$?
  \item Hvis en klinkekule slippes i $(1, 1, c - a - b)$, i hvilken retning vil den begynne
    å trille?
\end{enumerate}
\end{oppgave}
\begin{enumerate}
  \item
  \visible<2-4>{Høyden til $z$ stiger mest i retningen til gradienten $\nabla z(1,1)$ i
  $(1,1)$.}
\visible<3-4>{
  \begin{equation*}
    \nabla z(1,1)
    = (-2ax, -2by)\mid_{(1,1)}
    \ = 2 (-a, -b)
  \end{equation*}
  %
  Slik at høyden stiger mest i retning $(-a,-b)$.
}
\visible<4>{\item Klinkekulen vil trille i den retningen høyden avtar raskest,
altså i retning $-\nabla z(1,1)$ med andre ord $(a,b)$.}
\end{enumerate}

\end{frame}
 
% \begin{frame}
  \begin{tikzpicture}
    \begin{axis}[ xbar=0pt, /pgf/bar shift=0pt, legend style={ legend columns=4,
        at={(xticklabel cs:0.5)}, anchor=north, draw=none }, ytick={0,...,15},
      ytick style={draw=none},% <- added
      axis y line*=none, axis x line*=bottom, tick label
      style={font=\footnotesize}, legend style={font=\footnotesize}, label
      style={font=\footnotesize}, xtick style={draw=none},% <- added
      xtick={0,1,...,12}, width=.9\textwidth, bar width=3mm, y dir = reverse,
      xmin=0, xmax=13, area legend,
      y=5mm, enlarge y limits={abs=0.625},
      style={text=black}, every axis plot/.append style={fill},
      nodes near coords, nodes near coords,
      yticklabels={%
        {\topicref{Grenseverdier}},
        {\topicref{Gradient}},
        {\topicref{Epsilon-delta}},
        {\topicref{Kjerneregelen}},
        {\topicref{Taylor-approksimasjon}},
        {\topicref{Tangenter}},
        {\topicref{Kritiske-punkter}},
        {\topicref{Optimering}},
        {\topicref{Linjeintegral}},
        {\topicref{Konservative-vektorfelt}},
        {\topicref{Dobbelintegraler}},
        {\topicref{Integrasjonsrekkefolge}},
        {\topicref{Trippelintegral}},
        {\topicref{Greens-teorem}},
        {\topicref{Divergensteoremet}},
        {\topicref{Stokes-teorem}}}]
      \addplot[fill=black] coordinates {(8,0)};
      \addplot[fill=black] coordinates {(4,1)};
      \addplot[fill=red] coordinates {(1,2)};
      \addplot[fill=black] coordinates {(7,3)};
      \addplot[fill=black] coordinates {(1,4)};
      \addplot[fill=black] coordinates {(7,5)};
      \addplot[fill=black] coordinates {(11,6)};
      \addplot[fill=black] coordinates {(4,7)};
      \addplot[fill=black] coordinates {(3,8)};
      \addplot[fill=black] coordinates {(8,9)};
      \addplot[fill=black] coordinates {(7,10)};
      \addplot[fill=black] coordinates {(3,11)};
      \addplot[fill=black] coordinates {(4,12)};
      \addplot[fill=black] coordinates {(4,13)};
      \addplot[fill=black] coordinates {(8,14)};
      \addplot[fill=black] coordinates {(6,15)};
    \end{axis}
  \end{tikzpicture}
\end{frame}

\begin{frame}
  \subsection{Epsilon-Delta}\label{subsec:Epsilon-delta}
  \frametitle{Epsilon-Delta}\centerline{
  \only<1>{\includegraphics[scale=0.95]{../img/epsilon-delta}}%
  \only<2>{\includegraphics[scale=0.95]{../img/epsilon-delta-1}}
  \only<3>{\includegraphics[scale=0.95]{../img/epsilon-delta-2}}
  \only<4>{\includegraphics[scale=0.95]{../img/epsilon-delta-3}}}
[Vi sier at $f$ er \emph{kontinuerlig} i $\A$ dersom det for \only<1-3>{alle}<4>{\emph{alle}} $\varepsilon >
0$ eksisterer en $\delta > 0$ slik at
%
$
  \norm{f(\X) - f(\A)} < \varepsilon$ når $\norm{\X - \A} < \delta
  $.]\\
  Intuisjon: uansett hvor nærme vi er $\A$ kan vi alltid komme \emph{litt} nærmere.
\end{frame}

\begin{frame}
  \frametitle{Epsilon-Delta}
  \begin{itemize}
    \item Kladd: Forenkle $\norm{f(\X) - f(\Y)}$ til du oppnår ulikheten
      $\norm{f(\X)-f(\Y)} \leq K \norm{\X - \Y}$.
    \item Bevis: Skriv opp definisjonen ``ønsker å vise at for alle
      $\varepsilon>0$ eksisterer det en $\delta>0$ slik at''
      \begin{equation*}
    \norm{\X - \Y} < \delta \ \Longrightarrow \ \norm{f(\X) - f(\Y)} < \varepsilon.
  \end{equation*}
    \item Bevis: Velg $\delta = \varepsilon/K$. Da er
      \begin{align*}
        \norm{f(\X) - f(\Y)} \leq K \norm{\X - \Y} < K \delta \leq K(\varepsilon/K) = \varepsilon
      \end{align*}
    \item Litt som induksjon vi antar at $\norm{\X-\Y}<\delta$ stemmer (induksjonshypotesen).
  \end{itemize}
\end{frame}

\begin{frame}
  \begin{oppgave}{V2014, Oppgave 8}
    La funksjonen $f\colon A\subset \R^n \to \R^m$ oppfylle
    %
    $
      \norm{f(\X) - f(\Y)} \leq K \norm{\X - \Y}^{\alpha}
    $
    %
    for alle $x$, $y$ i $A$ hvor $K$ og $\alpha$ er positive. Vis at $f$ er en
    kontinuerlig funksjon.
  \end{oppgave}

\only<2>{\textbf{Kladd:} Antagelsen er at $\norm{\X - \Y}<\delta$. Slik at
%
\begin{equation*}
  \norm{f(\X) - f(\Y)}
  \leq K \norm{\X - \Y}^{\alpha}
  < K \delta^{\alpha}
\end{equation*}}
%
\visible<3->{%
\begin{proof}
  \visible<3->{Ønsker å vise at $f$ er kontinuerlig, med andre ord ønsker å vise at for alle
  $\varepsilon>0$ eksisterer det en $\delta>0$ slik at
  %
  \begin{equation*}
    \norm{\X - \Y} < \delta \ \Longrightarrow \ \norm{f(\X) - f(\Y)} < \varepsilon.
  \end{equation*}}%
%
\visible<4->{
  Velg $\delta = (\varepsilon/K)^{1/\alpha}$ da er
  %
  \begin{align*}
  \norm{f(\X) - f(\Y)}
    \leq K \norm{\X - \Y}^{\alpha}
    \visible<5->{%
    < K \delta^{\alpha}}
    \visible<6->{
    = K \left( \frac{\varepsilon^{1/\alpha}}{K^{1/\alpha}} \right)^{\alpha}
    = \varepsilon}
  \end{align*}}
  %
\visible<6->{som var det vi ønsket å vise.}
\end{proof}%
}
\end{frame}


%%% Local Variables:
%%% mode: latex
%%% TeX-master: "main"
%%% End:
           % Done
% \begin{frame}
  \section{Divergens og curl}
  \frametitle{Divergens}

\begin{itemize}
  \item Et vektorfelt kan betraktes som bevegelse av væsker eller gasser
  \item Divergens er \emph{endring} i fluks. Altså en operator som tar inn et vektorfelt, og gir ut en
    skalar som viser hvor mye væsketettheten endrer seg i ett punkt.
  \item Formelen for divergens er
    %
    \begin{equation*}
      \div \F = \nabla \cdot \F
              = \diffp{{F_1}}{x} + 
              \diffp{{F_2}}{y} +
              \cdots 
    \end{equation*}
    %
    Hvor $\F = (F_1, F_2, \ldots, F_n)$ er komponentene til $\F \colon \R^m \to \R^n$.
\end{itemize}

\begin{figure}
  \centering
  \begin{minipage}{.30\textwidth}
    \centering
  \includegraphics[scale=1.5]{../img/div-positive}
  \caption{$\nabla \cdot \F > 0 $}
\end{minipage}\hfill
\begin{minipage}{.30\textwidth}
    \centering
  \includegraphics[scale=1.5]{../img/div-negative}
  \caption{$\nabla \cdot \F <0 $}
\end{minipage}
\begin{minipage}{.30\textwidth}
    \centering
  \includegraphics[scale=1.5]{../img/div-null}
  \caption{$\nabla \cdot \F = 0$}
\end{minipage}
\end{figure}
\end{frame}

\begin{frame}
  \frametitle{Curl}
  \begin{itemize}
  \item Curlen viser hvor mye og i hvilken retning et
      vektorfelt “roterer”.
      \item Curlen er en vektor som peker i rotasjonsaksen og er proposjonal med rotasjonshastigheten.
    \item Dersom vektorfeltet roterer mot klokken sier vi at kurlen er positiv,
mot klokken er den negativ, mens dersom vektorfeltet er rotasjonsfritt er kurlen
      null.
    \item $\curl \F = \nabla \times \F =
      \begin{vmatrix} \I & \J & \K \\ \frac{\partial}{\partial x} &
        \frac{\partial}{\partial y} & \frac{\partial}{\partial z}\\ F_1 & F_2  & F_3
      \end{vmatrix} \ $ og $ \ \2d-curl \F = \diffp{{F_2}}{x} - \diffp{{F_1}}{y} $.
  \end{itemize}
  \begin{center}
      \includegraphics[scale=0.5]{../img/curl-negative.png}
\end{center}
\end{frame}

\begin{frame}
  \frametitle{Curl}
  \begin{figure}
  \centering
  \begin{minipage}{.30\textwidth}
    \centering
  \includegraphics[scale=0.75]{../img/rotation-2}
\end{minipage}
\begin{minipage}{.30\textwidth}
  \onslide<2->{%
  \begin{equation*}
    \phantom{1} \hspace{2cm} \2d-curl \F = \diffp{{F_2}}{x} - \diffp{{F_1}}{y}
  \end{equation*}}
\end{minipage}\hspace{2cm}
\end{figure}
\onslide<3->{
  \begin{align*}
    \label{eq:curl}
    \curl \F & = \nabla \times \F \\
    & = \left( \diffp{{F_3}}{y} - \diffp{{F_2}}{z} \right) \I
    + \left( \diffp{{F_1}}{z} - \diffp{{F_3}}{x} \right) \J
    + \left( \diffp{{F_2}}{x} - \diffp{{F_1}}{y} \right) \K.
  \end{align*}%
}
\end{frame}

\begin{frame}
  \begin{oppgave}{V2013, Oppgave 4}
    Vis sier at en funksjon er $C^2$ om alle andreordens partiellderiverte er
    kontinuerlige. Vis at om $f$ er en $C^2$ funksjon, så er $\curl\bigl( \nabla
    f\bigr) = \vek{0}$. Si klart fra hvordan du bruker at funksjonen er $C^2$. 
  \end{oppgave}
  \only<1>{%
  \begin{equation*}
    \curl \F =
    \begin{vmatrix}
      \I & \J & \K \\
      \frac{\partial}{\partial x} & \frac{\partial}{\partial y} & \frac{\partial}{\partial z}\\
      F_1 & F_2  & F_3
    \end{vmatrix}
     \qquad \text{og} \qquad
     \nabla f = \left( \diffp{f}{x}, \diffp{f}{y}, \diffp{f}{z} \right)
   \end{equation*}} \only<2->{%
 \begin{align*}
   \curl \bigl( \nabla f \bigr)
   \only<1-6>{& =
     \begin{vmatrix} \only<2>{\I} \only<3>{\circled{\I}} \only<6->{\I}
       & \only<2>{\J} \only<4>{\circled{\J}} \only<6->{\J}
       & \only<2>{\K} \only<5>{\circled{\K}} \only<6->{\K} \\
         \only<2>{\frac{\partial}{\partial x}} \only<4->{\frac{\partial}{\partial x}}
       & \only<2-3>{\frac{\partial}{\partial y}} \only<5->{\frac{\partial}{\partial y}}
       & \only<2-4>{\frac{\partial}{\partial y}} \only<6->{\frac{\partial}{\partial y}}\\
         \only<2>{\frac{\partial f}{\partial x}} \only<4->{\frac{\partial f}{\partial x}}
       & \only<2-3>{\frac{\partial f}{\partial y}} \only<5->{\frac{\partial f}{\partial y}}
       & \only<2-4>{\frac{\partial f}{\partial z}} \only<6->{\frac{\partial f}{\partial z}}
    \end{vmatrix}}
        \only<7>{&}\only<3-7>{=\I \begin{vmatrix} \frac{\partial}{\partial y} & \frac{\partial f}{\partial z} \\
                  \frac{\partial}{\partial z} & \frac{\partial f}{\partial y} \end{vmatrix}}
        \only<4-7>{- \J \begin{vmatrix} \frac{\partial}{\partial x} & \frac{\partial f}{\partial z} \\
                  \frac{\partial}{\partial x} & \frac{\partial f}{\partial y} \end{vmatrix}}
        \only<5-7>{+ \K \begin{vmatrix} \tfrac{\partial}{\partial y} & \tfrac{\partial f}{\partial x} \\
            \tfrac{\partial}{\partial z} & \tfrac{\partial f}{\partial y} \end{vmatrix}\\}
   \only<7->{
    & = \I \left( \tfrac{\partial}{\partial y} \tfrac{\partial f}{\partial z}
      -           \tfrac{\partial}{\partial z} \tfrac{\partial f}{\partial y} \right)
      - \J \left( \tfrac{\partial}{\partial y} \tfrac{\partial f}{\partial z}
      -           \tfrac{\partial}{\partial z} \tfrac{\partial f}{\partial y} \right)
      + \K \left( \tfrac{\partial}{\partial y} \tfrac{\partial f}{\partial z}
      -           \tfrac{\partial}{\partial z} \tfrac{\partial f}{\partial y} \right)\\}
   \only<8->{
    & = \I \left( \tfrac{\partial^2f}{\partial y \partial z} - \tfrac{\partial^2f}{\partial z\partial y} \right)
      - \J \left( \tfrac{\partial^2f}{\partial y \partial z} - \tfrac{\partial^2f}{\partial z\partial y} \right)
      + \K \left( \tfrac{\partial^2f}{\partial y \partial z} - \tfrac{\partial^2f}{\partial z\partial y} \right)}
   \only<9->{ \intertext{Siden funksjonen er $C^2$ er alle de kryss-partiellderiverte like slik at}
    & = \I 0 - \J 0 + \K 0 = \vek{0}}
  \end{align*}
}
\end{frame}





%%% Local Variables:
%%% mode: latex
%%% TeX-master: "main"
%%% End:
             % Done 
% \begin{frame}
  \begin{tikzpicture}
    \begin{axis}[ xbar=0pt, /pgf/bar shift=0pt, legend style={ legend columns=4,
        at={(xticklabel cs:0.5)}, anchor=north, draw=none }, ytick={0,...,15},
      ytick style={draw=none},% <- added
      axis y line*=none, axis x line*=bottom, tick label
      style={font=\footnotesize}, legend style={font=\footnotesize}, label
      style={font=\footnotesize}, xtick style={draw=none},% <- added
      xtick={0,1,...,12}, width=.9\textwidth, bar width=3mm, y dir = reverse,
      xmin=0, xmax=13, area legend,
      y=5mm, enlarge y limits={abs=0.625},
      style={text=black}, every axis plot/.append style={fill},
      nodes near coords, nodes near coords,
      yticklabels={%
        {\topicref{Grenseverdier}},
        {\topicref{Gradient}},
        {\topicref{Epsilon-delta}},
        {\topicref{Kjerneregelen}},
        {\topicref{Taylor-approksimasjon}},
        {\topicref{Tangenter}},
        {\topicref{Kritiske-punkter}},
        {\topicref{Optimering}},
        {\topicref{Linjeintegral}},
        {\topicref{Konservative-vektorfelt}},
        {\topicref{Dobbelintegraler}},
        {\topicref{Integrasjonsrekkefolge}},
        {\topicref{Trippelintegral}},
        {\topicref{Greens-teorem}},
        {\topicref{Divergensteoremet}},
        {\topicref{Stokes-teorem}}}]
      \addplot[fill=gray] coordinates {(8,0)};
      \addplot[fill=gray] coordinates {(4,1)};
      \addplot[fill=gray] coordinates {(1,2)};
      \addplot[fill=dgreen] coordinates {(7,3)};
      \addplot[fill=black] coordinates {(1,4)};
      \addplot[fill=black] coordinates {(7,5)};
      \addplot[fill=black] coordinates {(11,6)};
      \addplot[fill=black] coordinates {(4,7)};
      \addplot[fill=black] coordinates {(3,8)};
      \addplot[fill=black] coordinates {(8,9)};
      \addplot[fill=black] coordinates {(7,10)};
      \addplot[fill=black] coordinates {(3,11)};
      \addplot[fill=black] coordinates {(4,12)};
      \addplot[fill=black] coordinates {(4,13)};
      \addplot[fill=black] coordinates {(8,14)};
      \addplot[fill=black] coordinates {(6,15)};
    \end{axis}
  \end{tikzpicture}
\end{frame}

\begin{frame}
  \subsection{Kjerneregelen}\label{subsec:Kjerneregelen}
  \frametitle{Kjerneregelen}
  Anta vi skal derivere følgende funksjon
  %
  \begin{equation*}
    f(x,y) = \sin(x^2 - y^2) + \cos(y^2 - x^2)
  \end{equation*}
  %
  og ønsker å regne ut $x\tfrac{\partial f}{\partial y}+y\tfrac{\partial
    f}{\partial x}$.
  %
  \begin{align*}
    \diffp{f}{x}
    \only<1>{& = \diffp{}{x} \left( \sin(x^2 - y^2) \right) +
                 \diffp{}{x} \left( \cos(y^2 - y^2) \right)\\}
    \only<2>{& = \diffp{}{x} \left( x^2 - y^2\right) \cos(x^2+y^2) -
                 \diffp{}{x} \left( y^2 - x^2\right) \sin(y^2 - x^2)\\}
    \only<3>{& = \phantom{-}2x \cos(x^2+y^2) + 2x \sin(y^2 - x^2)\\}
      \diffp{f}{y}
    \only<1>{& = \diffp{}{y} \left( \sin(x^2 - y^2) \right) +
                 \diffp{}{y} \left( \cos(y^2 - x^2) \right)\\}
    \only<2>{& = \diffp{}{y} \left( x^2 - y^2 \right) \cos(x^2+y^2) -
                 \diffp{}{y} \left( y^2 - x^2 \right) \sin(y^2 - x^2)\\}
    \only<3>{& = -2y \cos(x^2+y^2) - 2y \sin(x^2 - y^2)}
  \end{align*}
  \only<3>{Ved inspeksjon ser vi at $x\tfrac{\partial f}{\partial y}+y\tfrac{\partial
    f}{\partial x} = 0$. Men gjelder dette alltid?}
\end{frame}

\begin{frame}
  \begin{oppgave}{K2014, Oppgave 2}
    Gitt $z = f (x, y)$ der $f$ er en deriverbar funksjon og
    $x = u^2 - v^2$ , $y = v^2 - u^2$. Vis at
    %
    \begin{equation*}
      u \diffp{z}{v} + v \diffp{z}{u} = 0\,.
    \end{equation*}
  \end{oppgave}
  Bruker kjerneregelen på $z = f(x,y) = f(u^2 - v^2, v^2 - u^2)$.
  %
  \begin{align*}
    \diffp{z}{u}
    \only<1>{& = \diffp{}{u} \left( f(x(u,v), y(u,v)) \right) \\}
    \only<2>{& = \diffp{x}{u}\diffp{f}{x} + \diffp{y}{u}\diffp{f}{y}\\}
    \only<3>{& = \phantom{-}2u\diffp{f}{x} - 2u\diffp{f}{y}\\}
      \diffp{z}{v}
    \only<1>{& = \diffp{}{v} \left( f(x(u,v), y(u,v)) \right)\\}
    \only<2>{& = \diffp{x}{v}\diffp{f}{x} + \diffp{y}{v}\diffp{f}{y}\\}
    \only<3>{& = -2v\diffp{f}{x} + 2v\diffp{f}{y}}
  \end{align*}
  \only<3>{Ved inspeksjon ser vi at $u\tfrac{\partial f}{\partial v}+v\tfrac{\partial
    f}{\partial u} = 0$ for alle deriverbare funksjoner $z=f(x,y)$.}
\end{frame}



%%% Local Variables:
%%% mode: latex
%%% TeX-master: "main"
%%% End:
           % Done
% \begin{frame}
  \begin{tikzpicture}
    \begin{axis}[ xbar=0pt, /pgf/bar shift=0pt, legend style={ legend columns=4,
        at={(xticklabel cs:0.5)}, anchor=north, draw=none }, ytick={0,...,15},
      ytick style={draw=none},% <- added
      axis y line*=none, axis x line*=bottom, tick label
      style={font=\footnotesize}, legend style={font=\footnotesize}, label
      style={font=\footnotesize}, xtick style={draw=none},% <- added
      xtick={0,1,...,12}, width=.9\textwidth, bar width=3mm, y dir = reverse,
      xmin=0, xmax=13, area legend,
      y=5mm, enlarge y limits={abs=0.625},
      style={text=black}, every axis plot/.append style={fill},
      nodes near coords, nodes near coords,
      yticklabels={%
        {\topicref{Grenseverdier}},
        {\topicref{Gradient}},
        {\topicref{Epsilon-delta}},
        {\topicref{Kjerneregelen}},
        {\topicref{Taylor-approksimasjon}},
        {\topicref{Tangenter}},
        {\topicref{Kritiske punkter}},
        {\topicref{Optimering}},
        {\topicref{Linjeintegral}},
        {\topicref{Konservative-vektorfelt}},
        {\topicref{Dobbelintegraler}},
        {\topicref{Integrasjonsrekkefolge}},
        {\topicref{Trippelintegral}},
        {\topicref{Greens-teorem}},
        {\topicref{Divergensteoremet}},
        {\topicref{Stokes-teorem}}}]
      \addplot[fill=black] coordinates {(8,0)};
      \addplot[fill=black] coordinates {(4,1)};
      \addplot[fill=black] coordinates {(1,2)};
      \addplot[fill=black] coordinates {(7,3)};
      \addplot[fill=red] coordinates {(1,4)};
      \addplot[fill=black] coordinates {(7,5)};
      \addplot[fill=black] coordinates {(11,6)};
      \addplot[fill=black] coordinates {(4,7)};
      \addplot[fill=black] coordinates {(3,8)};
      \addplot[fill=black] coordinates {(8,9)};
      \addplot[fill=black] coordinates {(7,10)};
      \addplot[fill=black] coordinates {(3,11)};
      \addplot[fill=black] coordinates {(4,12)};
      \addplot[fill=black] coordinates {(4,13)};
      \addplot[fill=black] coordinates {(8,14)};
      \addplot[fill=black] coordinates {(6,15)};
    \end{axis}
  \end{tikzpicture}
\end{frame}

\begin{frame}
  \subsection{Taylor-approksimasjon}\label{subsec:Taylor-approksimasjon}
  \frametitle{Taylor-approksimasjon}
  Some more text here
\end{frame}

\begin{frame}
  \begin{oppgave}{K2016, Oppgave 3}
    La $f(x,y) = x^4 + x^3 + y^2 + xy$. Finn andreordens Taylor-approksimasjon
    av $f$ i $(x_0, y_0) = (1,2)$.
  \end{oppgave} \visible<2->{\textbf{Steg 1:} Regn ut de partiellderiverte
  \begin{align*}
    & f_x(x,y) = 4x^3 + 3x^2 + y
      && \Rightarrow && f_x(1,2) = 9 \\
    & f_{xx}(x,y) = 12x^2 + 6x
      && \Rightarrow && f_{xx}(1,2) = 18 \\
    & f_{y}(x,y) = 2y + x
      && \Rightarrow && f_{y}(1,2) = 5 \\
    & f_{yy}(x, y) = 2
      && \Rightarrow && f_{yy}(1,2) = 2 \\
    & f_{xy}(x,y) = 1
      && \Rightarrow && f_{xy}(x,y) = 1 \\
    & f_{yx}(x,y) = 1
      && \Rightarrow && f_{yx}(x,y) = 1 \\
  \end{align*}}
\end{frame}

\begin{frame}
  \begin{oppgave}{K2016, Oppgave 3}
    La $f(x,y) = x^4 + x^3 + y^2 + xy$. Finn andreordens Taylor-approksimasjon
    av $f$ i $(x_0, y_0) = (1,2)$.
  \end{oppgave}
  \textbf{Steg 2:} Sett inn $\X^0 = (x_1^0,x_2^0)=(1,2)$ og $\X =
    (x_1, x_0) = (x,y)$ i formel.
    \begin{align*}
      T_2(\X) & = 
    \only<1-2>{f(\X^0)
               + \sum_{i=1}^{n} \only<1>{\diffp{f}{{x_i}}}%
                              \only<2>{f_{x_i}}(\X^0)(x_i - x_i^0)%
               + \frac{1}{2} \sum_{i,j=1}^{n} \only<1>{\diffp{f}{{x_i}{x_j}}}%
                                            \only<2>{f_{{x_i}{x_j}}}(\X^0)(x_i-x_i^0)(x_j - x_j^0)} 
    \only<3->{\underbrace{f(\X^0)}_{(1)}
      + \underbrace{\sum_{i=1}^{n} f_{{x_i}} (\X^0)(x_i - x_i^0)}_{(2)}
      + \underbrace{\frac{1}{2} \sum_{i,j=1}^{n} f_{{x_i}{x_j}}(\X^0)(x_i-x_i^0)(x_j - x_j^0)}_{(3)}} \\
    \only<5-14>{& = (8)} \only<10-14>{+ (9x + 5y - 19)} \only<14>{+ (9 x^2 + x y - 20 x + y^2 - 5 y + 15)}
    \only<15->{& = 4 - 11x + 9 x^2 + y^2 + xy }
  \end{align*}
  \begin{align*}
    \only<4-5>{(1)
    \mid f(\X^0) =} \only<4>{f(1,2)} \only<5>{8}
      \only<6-10>{(2)
    \left|\,\only<6>{\sum_{i=1}^{n} f_{{x_i}} (\X^0)(x_i - x_i^0)}
         \only<7->{\sum_{i=1}^{2} f_{{x_i}} (1,2)(x_i - x_i^0)} \right.
      \only<8>{= f_x(1,2) \cdot (x-1) + f_y(1,2) \cdot (y-2)} 
      \only<9>{= 9 (x-1) + 5(y-2)}
      \only<10>{= 9x + 5y - 19}}
    \only<11-14>{(3)
    & \left|\,
      \frac{1}{2}\sum_{i=1}^2 \sum_{j=1}^2 f_{{x_i}{x_j}}(1,2)(x_i - x_i^0) (x_j - x_j^0)\right. \\
      & = \only<11-12>{\frac{1}{2}f_{xx}(1,2)(x-1)^2 + \only<11>{\frac{1}{2}}f_{xy}(1,2)(x-1)(y-2) + \only<11>{\\
    & \hspace{4cm} \frac{1}{2}f_{yx}(1,2)(y-2)(x-1) +} \frac{1}{2}f_{yy}(1,2)(y-2)^2}}
      \only<13>{\frac{18}{2}(x-1) + 1 \cdot (x-1)(y-2) + \frac{2}{2}(y-2)^2}
      \only<14>{9 x^2 + x y - 20 x + y^2 - 5 y + 15}
  \end{align*}
\end{frame}

\begin{frame}
  \begin{oppgave}{K2016, Oppgave 3}
    La $f(x,y) = x^4 + x^3 + y^2 + xy$. Finn andreordens Taylor-approksimasjon
    av $f$ i $(x_0, y_0) = (1,2)$.
  \end{oppgave}
  \includegraphics[width=0.5\textwidth]{../img/taylor-zx}%
  \includegraphics[width=0.5\textwidth]{../img/taylor-zy}
\end{frame}

%%% Local Variables:
%%% mode: latex
%%% TeX-master: "main"
%%% End:
   % Done
\begin{frame}
  \begin{tikzpicture}
    \begin{axis}[ xbar=0pt, /pgf/bar shift=0pt, legend style={ legend columns=4,
        at={(xticklabel cs:0.5)}, anchor=north, draw=none }, ytick={0,...,15},
      ytick style={draw=none},% <- added
      axis y line*=none, axis x line*=bottom, tick label
      style={font=\footnotesize}, legend style={font=\footnotesize}, label
      style={font=\footnotesize}, xtick style={draw=none},% <- added
      xtick={0,1,...,12}, width=.9\textwidth, bar width=3mm, y dir = reverse,
      xmin=0, xmax=13, area legend,
      y=5mm, enlarge y limits={abs=0.625},
      style={text=black}, every axis plot/.append style={fill},
      nodes near coords, nodes near coords,
      yticklabels={%
        {\topicref{Grenseverdier}},
        {\topicref{Gradient}},
        {\topicref{Epsilon-delta}},
        {\topicref{Kjerneregelen}},
        {\topicref{Taylor-approksimasjon}},
        {\topicref{Tangenter}},
        {\topicref{Kritiske-punkter}},
        {\topicref{Optimering}},
        {\topicref{Linjeintegral}},
        {\topicref{Konservative-vektorfelt}},
        {\topicref{Dobbelintegraler}},
        {\topicref{Integrasjonsrekkefolge}},
        {\topicref{Trippelintegral}},
        {\topicref{Greens-teorem}},
        {\topicref{Divergensteoremet}},
        {\topicref{Stokes-teorem}}}]
      \addplot[fill=black] coordinates {(8,0)};
      \addplot[fill=black] coordinates {(4,1)};
      \addplot[fill=black] coordinates {(1,2)};
      \addplot[fill=black] coordinates {(7,3)};
      \addplot[fill=black] coordinates {(1,4)};
      \addplot[fill=red] coordinates {(7,5)};
      \addplot[fill=black] coordinates {(11,6)};
      \addplot[fill=black] coordinates {(4,7)};
      \addplot[fill=black] coordinates {(3,8)};
      \addplot[fill=black] coordinates {(8,9)};
      \addplot[fill=black] coordinates {(7,10)};
      \addplot[fill=black] coordinates {(3,11)};
      \addplot[fill=black] coordinates {(4,12)};
      \addplot[fill=black] coordinates {(4,13)};
      \addplot[fill=black] coordinates {(8,14)};
      \addplot[fill=black] coordinates {(6,15)};
    \end{axis}
  \end{tikzpicture}
\end{frame}

\begin{frame}
  \subsection{Tangenter}\label{subsec:Tangenter}
  \only<1>{\frametitle{Tangenter}}
  \begin{oppgave}{Matte 2 -- Våren 2010, Oppgave 4b}
    La $f(x,y) = x^3 + y^3 - 3xy$. Skriv opp likningen for tangentplanet til
    grafen til funksjonen $z = f(x,y)$ i punktet $(2,2,4)$. 
  \end{oppgave}
  \textbf{Plan:}
  \begin{enumerate}
    \item Finne normalvektor ved hjelp av gradienten $g(x,y,z) = f(x,y) - z$.
\visible<2->{
      Gradienten er gitt som
      %
      \begin{align*}
        \nabla g(x,y,z)
        = (\only<2>{\diffp{g}{x}}\only<3>{\diffp{}{x}\left( x^3+y^3-3xy -z\right)}\only<4->{3x^2 - 3y},
          \only<2-4>{\diffp{g}{y}}\only<5>{\diffp{}{y}\left( x^3+y^3-3xy -z\right)}\only<6->{3y^2 - 3x},
          \only<2-6>{\diffp{g}{z}}\only<7>{\diffp{}{z}\left( x^3+y^3-3xy -z\right)}\only<8->{-1})
      \end{align*}
      %
      Siden gradienten peker i samme retning som normalvektorn setter vi
      $\vek{n} = \nabla g(2,2,4) = (6,6,-1)$.}
    \item Bruke den generelle likningen for et plan (normalvektor prikket med
      posisjonsvektor) $\vek{n} \cdot (\rr - \rr_0) = 0$.

      \only<9->{Hvor $\vek{\rr} = (x,y,z)$, $\rr_0=(2,2,4)$ og $\vek{n} = (6,6,-1)$
      %
      \begin{align*}
        \only<9>{\vek{n} \cdot (\rr - \rr_0) = 0}
        \only<10>{(6,6,-1) \cdot ( (x-2,y-2,z-4) = 0}
        \only<11>{6(x-2) + 6(y-2) - (z-4)= 0}
        \only<12>{z = 6x + 6y - 20.}
    \end{align*}}
  \end{enumerate}
\end{frame}

%%% Local Variables:
%%% mode: latex
%%% TeX-master: "main"
%%% End:
               % Done
\begin{frame}
  \begin{tikzpicture}
    \begin{axis}[ xbar=0pt, /pgf/bar shift=0pt, legend style={ legend columns=4,
        at={(xticklabel cs:0.5)}, anchor=north, draw=none }, ytick={0,...,15},
      ytick style={draw=none},% <- added
      axis y line*=none, axis x line*=bottom, tick label
      style={font=\footnotesize}, legend style={font=\footnotesize}, label
      style={font=\footnotesize}, xtick style={draw=none},% <- added
      xtick={0,1,...,12}, width=.9\textwidth, bar width=3mm, y dir = reverse,
      xmin=0, xmax=13, area legend,
      y=5mm, enlarge y limits={abs=0.625},
      style={text=black}, every axis plot/.append style={fill},
      nodes near coords, nodes near coords,
      yticklabels={%
        {\topicref{Grenseverdier}},
        {\topicref{Gradient}},
        {\topicref{Epsilon-delta}},
        {\topicref{Kjerneregelen}},
        {\topicref{Taylor-approksimasjon}},
        {\topicref{Tangenter}},
        {\topicref{Kritiske-punkter}},
        {\topicref{Optimering}},
        {\topicref{Linjeintegral}},
        {\topicref{Konservative-vektorfelt}},
        {\topicref{Dobbelintegraler}},
        {\topicref{Integrasjonsrekkefolge}},
        {\topicref{Trippelintegral}},
        {\topicref{Greens-teorem}},
        {\topicref{Divergensteoremet}},
        {\topicref{Stokes-teorem}}}]
      \addplot[fill=black] coordinates {(8,0)};
      \addplot[fill=black] coordinates {(4,1)};
      \addplot[fill=black] coordinates {(1,2)};
      \addplot[fill=black] coordinates {(7,3)};
      \addplot[fill=black] coordinates {(1,4)};
      \addplot[fill=black] coordinates {(7,5)};
      \addplot[fill=red] coordinates {(11,6)};
      \addplot[fill=black] coordinates {(4,7)};
      \addplot[fill=black] coordinates {(3,8)};
      \addplot[fill=black] coordinates {(8,9)};
      \addplot[fill=black] coordinates {(7,10)};
      \addplot[fill=black] coordinates {(3,11)};
      \addplot[fill=black] coordinates {(4,12)};
      \addplot[fill=black] coordinates {(4,13)};
      \addplot[fill=black] coordinates {(8,14)}; 
      \addplot[fill=black] coordinates {(6,15)};
    \end{axis}
  \end{tikzpicture}
\end{frame}

\begin{frame}
  \subsection{Kritiske punkter}\label{subsec:Kritiske-punkter}
  \only<1>{\frametitle{Kritiske punkter}}
  \begin{theorem}[Annenderivertetesten i to variable]
    La $\A$ være et punkt kritisk punkt ($\nabla f(\A) = 0$) for en funksjon
    $f(x,y)$ med kontinuerlige annenordens partiellderiverte. Definer $D(x,y)$
    som
    %
    \begin{equation*}
      D(x,y) :=
      \begin{vmatrix}
        f_{xx} & f_{xy} \\ f_{yx} & f_{yy}
      \end{vmatrix}
      = \underbrace{f_{xx}}_{A}\underbrace{f_{yy}}_{B} - \underbrace{(f_{xy})^2}_{C}
    \end{equation*}
    %
    \vspace{-0.75cm}
    \begin{enumerate}
      \item Hvis $D < 0$, så er $\A$ et saddelpunkt.
      \item Hvis $D > 0$ og $f_{xx}>0$, så er $\A$ et lokalt minimum.
      \item Hvis $D > 0$ og $f_{xx}<0$, så er $\A$ et lokalt maksimum.
    \end{enumerate}
    %
    Hvis $D = 0$ gir testen ingen konklusjon.
  \end{theorem}
  \visible<2->{
  \begin{intuisjon}
    \only<2>{Anta $f_{xx}f_{yy}<0$. Enten $f_{xx}<0$ og $f_{yy}>0$ eller
    $f_{yy}<0$ og $f_{xx}>0$ uansett så har funksjonen positiv krumning $\cup$ i
    den ene retningen og negativ krumning 
    i den andre $\cap$ $\Rightarrow$ åpenbart saddelpunkt (pringles).}
  \only<3>{Anta $f_{xx}f_{yy}>0$. Hvis $f_{xx}>0$ og $f_{yy}>0$ så har $f$
    positiv krumning $\cup$ omkring
    $\A$ $\Rightarrow$ minimum. (bolle) Hvis $f_{xx}<0$ og $f_{yy}<0$ så har $f$
    negativ krumning $\cap$ omkring $A$
omkring $\A$ $\Rightarrow$ maksimum. (øvre halvkule)}
\only<4>{Tilfellet $f_{xx}f_{yy} = 0$ behandles av leddet $f_{xy}^2$. Ligner funksjonen $f$ mer
på en (pringles) eller en (bolle/øvre halvkule)?}
  \end{intuisjon}}
\end{frame}

\begin{frame}
  \begin{oppgave}{K2015, Oppgave 4a}
    Funksjonen $f \colon \R^2 \to \R$ er gitt ved $f(x,y) = 2x^2 - x^4 + y^2$. Finn alle kritiske
    punkter til $f$, og avgjør om disse er lokale maksima, minima eller
    saddelpunkter. 
  \end{oppgave}%
  %
  \vspace{-0.5cm}
\begin{equation*}
  \La(x,y) := f(x,y) = 2x^2 - x^4 + y^2
\end{equation*}
%
\only<2->{De partiellderiverte blir
%
\begin{equation*}
  \diffp{\La}{x} = \only<2>{4x - 4x^3}\only<3->{4x(1-x^2)}
  \quad \text{og} \quad 
  \diffp{\La}{y} = 2y
\end{equation*}}
%
\only<4->{Likningsettet $\nabla \La = \vek{0}$ har løsninger $(0,0)$ og $(\pm
  1,0)$.} \visible<5->{Determinanten
  %
\begin{align*}
  D(x,y) = \only<5>{ f_{xx} f_{yy} - f_{xy}^2}
  \only<6>{ \diffp[2]{}{x}(2x^2 - x^4 + y^2) \diffp[2]{}{y}(2x^2 - x^4 + y^2) - \diffp{}{xy}(2x^2 - x^4 + y^2)}
  \only<7>{ \diffp{}{x}(4x-4x^2) \diffp{}{y}(2y) - \diffp{}{x}(2y)}
  \only<8->{ 8(1 - 3x^2)}\only<9->{\hspace{3.75cm}\\[-1cm]}
%
\only<9->{\intertext{Innsetning gir}\\[-0.8cm]
%
  D(0,0) \only<9>{= 8}\only<10->{>0}, \  f_{xx}(0,0) \only<9>{= 4}\only<10>{> 0} \ \ \text{og} \ \
  D(\pm 1, 0) \only<9>{= -16}\only<10->{<0}, \  f_{xx}(\pm 1,0) \only<9>{=-8}\only<10->{<0}}\\[-0.8cm]
\end{align*}
%
\only<9->{slik at $(0,0)$ er ett \only<10->{lokalt
    minimum}\only<9>{\underline{\phantom{lokalt minimum}}} og $(\pm 1,0)$ er \only<9>{\underline{\phantom{saddelpunkter}}}\only<10>{saddelpunkter}.}}
\end{frame}

\begin{frame}
  \begin{oppgave}{K2015, Oppgave 4b}
    Funksjonen $f \colon \R^2 \to \R$ er gitt ved $f(x,y) = 2x^2 - x^4 + y^2$.
    Finn største og minste verdi for $f$ på kurven $x^4 + y^2 = 4$.
  \end{oppgave}%
  %
  \vspace{-0.5cm}
  \begin{equation*}
    \La(x,y) := 2x^2 - x^4 + y^2 - \lambda (x^4 + y^2 - 4)
  \end{equation*}
%
  \only<2->{$f_{\text{min}} = \only<2-4>{?}\only<5->{0}$, $f_{\text{max}} =
    \only<2-7>{?}\only<8-9>{4}\only<10->{4 + \frac{1}{2}}$. De partiellderiverte blir
%
\begin{align*}
  & \diffp{\La}{x} = 4x - 4x^3 - 4\lambda x^3,
  &&\diffp{\La}{y} = \only<2>{2y - \lambda 2y}\only<3->{2y(1-\lambda)}
  \quad \text{og} \quad 
  &&\diffp{\La}{\lambda} = x^4 + y^2 - 4
%
\only<3>{\intertext{Fra $\La_y = 2y(1-\lambda)=0$ får vi at enten $y = 0$ eller
  $\lambda = 1$.}}
\visible<4->{\intertext{\underline{$\only<4-5>{y=0}\only<6->{\lambda=1}$:}}
  %
  &\diffp{\La}{x} = \only<4-6>{4x - 4x^3 - 4\only<4-5>{\lambda} x^3}\only<7->{4x(1-2x^2)},
  \quad
  &&\diffp{\La}{y} = 0
  \quad \text{og} \quad 
  &&\diffp{\La}{\lambda} = x^4 \only<6->{+ y^2} - 4}
\end{align*}}
%
  \only<4-5>{Slik at $x^4 = 4 \Rightarrow x^2 = 2$ og $f = 2\only<4>{x^2}\only<5>{\cdot 2}
    - \only<4>{x^4}\only<5>{4} + \only<4>{y^2}\only<5>{0^2} \only<5>{= 0}$.}
  \only<7-8>{Når $x = 0$ så gir $\La_\lambda$, $y^2 = 4$ slik at $f =
    2\only<7>{x^2}\only<8>{\cdot 0^2} - \only<7>{x^3}\only<8>{0^3} +
    \only<7>{y^2}\only<8>{4} \only<8>{= 4}$}
  \only<9-10>{Når $x^2 = \frac{1}{2} \Rightarrow x^4 = \frac{1}{4}$ og
    $y^2 = 4-x^4$ så
    $f = 2\only<9>{x^2} \only<10>{\frac{1}{2}} - \only<9>{x^4}\only<10>{\frac{1}{4}} +
    \only<9>{y^2}\only<10>{4 - \frac{1}{4}} \only<10>{= 4 + \frac{1}{2}}$.}
\end{frame}


%%% Local Variables:
%%% mode: latex
%%% TeX-master: "main"
%%% End:
        % Done
% \begin{frame}
  \begin{tikzpicture}
    \begin{axis}[ xbar=0pt, /pgf/bar shift=0pt, legend style={ legend columns=4,
        at={(xticklabel cs:0.5)}, anchor=north, draw=none }, ytick={0,...,15},
      ytick style={draw=none},% <- added
      axis y line*=none, axis x line*=bottom, tick label
      style={font=\footnotesize}, legend style={font=\footnotesize}, label
      style={font=\footnotesize}, xtick style={draw=none},% <- added
      xtick={0,1,...,12}, width=.9\textwidth, bar width=3mm, y dir = reverse,
      xmin=0, xmax=13, area legend,
      y=5mm, enlarge y limits={abs=0.625},
      style={text=black}, every axis plot/.append style={fill},
      nodes near coords, nodes near coords,
      yticklabels={%
        {\topicref{Grenseverdier}},
        {\topicref{Gradient}},
        {\topicref{Epsilon-delta}},
        {\topicref{Kjerneregelen}},
        {\topicref{Taylor-approksimasjon}},
        {\topicref{Tangenter}},
        {\topicref{Kritiske-punkter}},
        {\topicref{Optimering}},
        {\topicref{Linjeintegral}},
        {\topicref{Konservative-vektorfelt}},
        {\topicref{Dobbelintegraler}},
        {\topicref{Integrasjonsrekkefolge}},
        {\topicref{Trippelintegral}},
        {\topicref{Greens-teorem}},
        {\topicref{Divergensteoremet}},
        {\topicref{Stokes-teorem}}}]
      \addplot[fill=gray] coordinates {(8,0)};
      \addplot[fill=gray] coordinates {(4,1)};
      \addplot[fill=gray] coordinates {(1,2)};
      \addplot[fill=gray] coordinates {(7,3)};
      \addplot[fill=gray] coordinates {(1,4)};
      \addplot[fill=gray] coordinates {(7,5)};
      \addplot[fill=gray] coordinates {(11,6)};
      \addplot[fill=dgreen] coordinates {(4,7)};
      \addplot[fill=black] coordinates {(3,8)};
      \addplot[fill=black] coordinates {(8,9)};
      \addplot[fill=black] coordinates {(7,10)};
      \addplot[fill=black] coordinates {(3,11)};
      \addplot[fill=black] coordinates {(4,12)};
      \addplot[fill=black] coordinates {(4,13)};
      \addplot[fill=black] coordinates {(8,14)};
      \addplot[fill=black] coordinates {(6,15)};
    \end{axis}
  \end{tikzpicture}
\end{frame}

\begin{frame}
  \subsection{Optimering}\label{subsec:Optimering}
  \frametitle{Optimering}
  \begin{oppgave}{K2015, Oppgave 4a}
    Finn punktene på kuleflaten $x^2 + y^2 + z^2 = 4$ som er nærmeste og lengst
    fra punktet $(2,2,1)$.
  \end{oppgave}%
  \visible<1->{
    \only<1-3>{
    Avstanden mellom punktet $(x,y,z)$ og $(2,2,1)$ er
    %
    \begin{equation*}
      f(x,y,z) = (x-2)^2 + (y-2)^2 + (z-1)^2
    \end{equation*}
    %
    Vi ønsker å optimalisere denne avstanden under bi-bibetingelsen
    om at punktet vårt ligger på kuleflaten. Altså $g(x,y,z)=x^2+y^2+z^2-4$.
    %
    \begin{equation*}
      \nabla f = \lambda \nabla g
      \Leftrightarrow 2(x - 2, y - 2, z - 1) = 2 \lambda (x,y,z)
    \end{equation*}
    %
    \only<2>{
    Her har vi altså tre likninger med tre ukjente. Med løsning
    %
    \begin{equation*}
      x = \frac{2}{1 - \lambda} = y \quad \text{og} \quad z = \frac{1}{1-\lambda}
  \end{equation*}}
  \only<3>{ Innsatt i $g(x,y,z)$ får vi da
    \begin{equation*}
      \Bigl(\frac{1}{1-\lambda}\Bigr)^2(2^2 + 2^2 + 1) = 4
      \Rightarrow \frac{1}{1-\lambda} = \pm \frac{2}{3}
  \end{equation*}
  %
  slik at $x=y=\pm 4/3$, $z=\pm 2/3$ punktene $(4,4,2)/3$ og $(-4,-4,-2)/3$
  som
}}
\only<4-5>{
\begin{columns}[T] % align columns
\begin{column}{.47\textwidth}
  \visible<4-5>{
For å reise raskest mulig bort fra kulen
må vi reise normalt bort fra den. Dette
er retningen til normalvektoren
$\vek{n}=(2,2,1)$.
\includegraphics[width=0.75\textwidth]{normalvektor.png}}
\end{column}%
\hfill%
\begin{column}{.50\textwidth}
  \visible<5>{
  Finner skjæringspunktet mellom
  %
  \begin{equation*}
    \ell(t) = (0,0,0)+t(2,2,1) = t(2,2,1)
  \end{equation*}
  %
  og kuleflaten $x^2+y^2+z^2=4$. Dette gir
  %
  \begin{equation*}
    (2t)^2+  (2t)^2 + (t)^2 = 4
    \Rightarrow t = \pm \frac{2}{3}
  \end{equation*}
  %
  Leser kan selv teste at en får samme punkter
ved å sette $t$ inn i $\ell$.}
\end{column}%
\end{columns}
}
  }
\end{frame}



% Integrasjon
{%
  \setbeamercolor{background canvas}{bg=black}
  \setbeamercolor{frametitle}{fg=white}
  \begin{frame}
  \frametitle{\bfseries Integrasjon}
  \section{Integrasjon}
  \begin{center}
    \includegraphics[width=0.5\textwidth]{../img/3d-dobbelintegral}%
    \includegraphics[width=0.5\textwidth]{../img/massetetthet}
  \end{center}
  \begin{equation*}
    \color{white}
    \iint_D \int_0^{f(x,y)} 1 \dz \dd(x,y)
    \qquad  = \qquad  \iint_{D} f(x,y) \dd (x,y)
  \end{equation*}
  \end{frame}%
}

%%% Local Variables:
%%% mode: latex
%%% TeX-master: "main"
%%% End:
             % Done 
% \begin{frame}
  \begin{tikzpicture}
    \begin{axis}[ xbar=0pt, /pgf/bar shift=0pt, legend style={ legend columns=4,
        at={(xticklabel cs:0.5)}, anchor=north, draw=none }, ytick={0,...,15},
      ytick style={draw=none},% <- added
      axis y line*=none, axis x line*=bottom, tick label
      style={font=\footnotesize}, legend style={font=\footnotesize}, label
      style={font=\footnotesize}, xtick style={draw=none},% <- added
      xtick={0,1,...,12}, width=.9\textwidth, bar width=3mm, y dir = reverse,
      xmin=0, xmax=13, area legend,
      y=5mm, enlarge y limits={abs=0.625},
      style={text=black}, every axis plot/.append style={fill},
      nodes near coords, nodes near coords,
      yticklabels={%
        {\topicref{Grenseverdier}},
        {\topicref{Gradient}},
        {\topicref{Epsilon-delta}},
        {\topicref{Kjerneregelen}},
        {\topicref{Taylor-approksimasjon}},
        {\topicref{Tangenter}},
        {\topicref{Kritiske-punkter}},
        {\topicref{Optimering}},
        {\topicref{Linjeintegral}},
        {\topicref{Konservative-vektorfelt}},
        {\topicref{Dobbelintegraler}},
        {\topicref{Integrasjonsrekkefolge}},
        {\topicref{Trippelintegral}},
        {\topicref{Greens-teorem}},
        {\topicref{Divergensteoremet}},
        {\topicref{Stokes-teorem}}}]
      \addplot[fill=black] coordinates {(8,0)};
      \addplot[fill=black] coordinates {(4,1)};
      \addplot[fill=black] coordinates {(1,2)};
      \addplot[fill=black] coordinates {(7,3)};
      \addplot[fill=black] coordinates {(1,4)};
      \addplot[fill=black] coordinates {(7,5)};
      \addplot[fill=black] coordinates {(11,6)};
      \addplot[fill=black] coordinates {(4,7)};
      \addplot[fill=red] coordinates {(3,8)};
      \addplot[fill=black] coordinates {(8,9)};
      \addplot[fill=black] coordinates {(7,10)};
      \addplot[fill=black] coordinates {(3,11)};
      \addplot[fill=black] coordinates {(4,12)};
      \addplot[fill=black] coordinates {(4,13)};
      \addplot[fill=black] coordinates {(8,14)};
      \addplot[fill=black] coordinates {(6,15)};
    \end{axis}
  \end{tikzpicture}
\end{frame}

\begin{frame}
  \subsection{Linjeintegral}\label{subsec:Linjeintegral}
  \frametitle{Linjeintegral}
  \centerline{%
    \only<1>{\includegraphics[scale=0.95]{../img/flytskjema-linjeintegral}}%
    \only<2>{\includegraphics[scale=0.95]{../img/flytskjema-linjeintegral-1}}
  }
  $\partial D$ er en kurve slik at $\partial D \colon \rr(t), \ a \leq t \leq b$
  med $\rr(a)=A$ og $\rr(b) = B$. \\
  $D$ er området avgrenset av $\partial D$. Dersom $\F$ er konservativ er $\nabla f = \F$.
\end{frame}

\begin{frame}
  \begin{columns}
    \column{0.58\linewidth}
  \begin{oppgave}{K2016, Oppgave 6}
    La $\F(x,y) = (0, x)$. Hva er verdien av integralet
    %
    \begin{equation*}
      \int_C \F \cdot \dr
    \end{equation*}
    %
    der $C$ er en sirkel med radius $a > 0$? 
  \end{oppgave}
  
  \column{0.38\linewidth} \vspace{0.5cm}
  \only<1>{\includegraphics{../img/vektorfelt-linjeintegral-0}}
  \only<2>{\includegraphics{../img/vektorfelt-linjeintegral-1}}
  \only<3>{\includegraphics{../img/vektorfelt-linjeintegral-2}}
  \only<4>{\includegraphics{../img/vektorfelt-linjeintegral-3}}
  \only<5>{\includegraphics{../img/vektorfelt-linjeintegral-4}}
  \only<6->{\includegraphics{../img/vektorfelt-linjeintegral}}
  \end{columns}
    \begin{align*}
    \rr\phantom{'}(\theta)
    & = (x_0 + a \cos \only<1>{\theta}\only<2>{0}\only<3>{\frac{\pi}{2}}\only<4>{\pi}\only<5>{\frac{3\pi}{2}}\only<6->{\theta}) \I
      + (y_0 + a \sin \only<1>{\theta}\only<2>{0}\only<3>{\frac{\pi}{2}}\only<4>{\pi}\only<5>{\frac{3\pi}{2}}\only<6->{\theta}) \J \hfill \visible<8->{\\
  \rr'(\theta) & = \hspace{0.75cm} - a \sin \theta\phantom{)}\I + \hspace{0.8cm} (a \cos \theta) \J\\
    \F\bigl(\rr(\theta)\bigr) & = \hspace{1.85cm} 0 \phantom{)} \I + \hspace{-0.05cm} (x_0 + a \cos \theta) \J \\}
    \visible<7->{
    \int_C \F \cdot \dr
    & = \int_0^{2\pi} \F\bigl( \rr(\theta) \bigr) \rr'(\theta) \dT \\} \visible<9->{
      & = \int_0^{2\pi} ax_0 (\cos \theta) + a^2 (\cos \theta)^2 \dT } \visible<10->{
      = \int_0^{2\pi} \frac{a^2}{2} \dT 
      =  \pi a^2 }
    \end{align*}
  \end{frame}

  \begin{frame}
\frametitle{Trigonometri: like funksjoner}
    \begin{tikzpicture}
        \begin{axis}[
            Trigonometric Style,
            legend style={at={(0.68,1)},anchor=north east},
            legend style = {draw = none},
            legend style = {fill=none},
            xtick={
                -6.28318, -4.7123889, -3.14159, -1.5708,
                1.5708, 3.14159, 4.7123889, 6.28318
            },
            xticklabels={
                $-2\pi$, $-\frac{3\pi}{2}$, $-\pi$, $-\frac{\pi}{2}$,
                $\frac{\pi}{2}$, $\pi$, $\frac{3\pi}{2}$, $2\pi$
            }
        ]
        \addplot [mark=none, ultra thick, red] {cos(deg(x))^2};
        \addplot [mark=none, ultra thick, blue] {sin(deg(x))^2};
        \legend{$(\cos x)^2$,$(\sin x)^2$}
        \end{axis}
    \end{tikzpicture}
    \begin{theorem}
    Det å integrere $(\cos x)^2$ eller $(\sin x)^2$ over intervaler som er multiplum av $\pi/2$ er det samme som å integrere $1/2$ over samme intervalet:
        \begin{align*}
              \int_{m\pi/2}^{n\pi/2} (\sin x)^2 \mathrm{d}x
            = \int_{m\pi/2}^{n\pi/2}  (\cos x)^2 \mathrm{d}x
            = \frac{1}{2} \int_{m\pi/2}^{n\pi/2}  \mathrm{d}x \qquad m,n \in \mathbb{Z}
    \end{align*}
    \end{theorem}
  \end{frame}

  \begin{frame}
    \begin{tikzpicture}
        \begin{axis}[
            Trigonometric Style,
            legend style={at={(0.68,1)},anchor=north east},
            legend style = {draw = none},
            legend style = {fill=none},
            xtick={
                -6.28318, -4.7123889, -3.14159, -1.5708,
                1.5708, 3.14159, 4.7123889, 6.28318
            },
            xticklabels={
                $-2\pi$, $-\frac{3\pi}{2}$, $-\pi$, $-\frac{\pi}{2}$,
                $\frac{\pi}{2}$, $\pi$, $\frac{3\pi}{2}$, $2\pi$
            }
        ]
        \addplot [mark=none, ultra thick, red] {cos(deg(x))^2};
        \addplot [mark=none, ultra thick, blue] {sin(deg(x))^2};
        \legend{$(\cos x)^2$,$(\sin x)^2$}
        \end{axis}
      \end{tikzpicture}
      \begin{align}
        &\int_0^{\pi/2} (\sin x)^2 \dx
        \only<1>{ \overset{u \mapsto x - \pi/2}{=} -\int_{\pi/2}^0 (\sin(\pi/2 - u))^2 \du}
        \only<2->{= \int_0^{\pi/2} (\cos x)^2 \dx }
        \only<3>{\\ & (\cos x)^2 + (\sin x)^2 = 1}\only<4>{\\ & \int_0^{\pi/2} (\sin x)^2 + (\cos x)^2 \dx = \int_0^{\pi/2} 1 \dx}\only<5->{\\ & \int_0^{\pi/2} (\sin x)^2\dx + \int_0^{\pi/2}(\cos x)^2 \dx = \int_0^{\pi/2} 1 \dx}
                                                                                                                          \visible<6->{\\  & \hspace{3cm}\Downarrow \notag \\
        &\int_0^{\pi/2} (\sin x)^2 \dx = \int_0^{\pi/2} (\cos x)^2 \dx  = \int_0^{\pi/2} \frac{\dx}{2}}
      \end{align}
  \end{frame}

%%% Local Variables:
%%% mode: latex
%%% TeX-master: "main"
%%% End:
           % Done 
% \begin{frame}
  \begin{tikzpicture}
    \begin{axis}[ xbar=0pt, /pgf/bar shift=0pt, legend style={ legend columns=4,
        at={(xticklabel cs:0.5)}, anchor=north, draw=none }, ytick={0,...,15},
      ytick style={draw=none},% <- added
      axis y line*=none, axis x line*=bottom, tick label
      style={font=\footnotesize}, legend style={font=\footnotesize}, label
      style={font=\footnotesize}, xtick style={draw=none},% <- added
      xtick={0,1,...,12}, width=.9\textwidth, bar width=3mm, y dir = reverse,
      xmin=0, xmax=13, area legend,
      y=5mm, enlarge y limits={abs=0.625},
      style={text=black}, every axis plot/.append style={fill},
      nodes near coords, nodes near coords,
      yticklabels={%
        {\topicref{Grenseverdier}},
        {\topicref{Gradient}},
        {\topicref{Epsilon-delta}},
        {\topicref{Kjerneregelen}},
        {\topicref{Taylor-approksimasjon}},
        {\topicref{Tangenter}},
        {\topicref{Kritiske-punkter}},
        {\topicref{Optimering}},
        {\topicref{Linjeintegral}},
        {\topicref{Konservative-vektorfelt}},
        {\topicref{Dobbelintegraler}},
        {\topicref{Integrasjonsrekkefolge}},
        {\topicref{Trippelintegral}},
        {\topicref{Greens-teorem}},
        {\topicref{Divergensteoremet}},
        {\topicref{Stokes-teorem}}}]
      \addplot[fill=gray] coordinates {(8,0)};
      \addplot[fill=gray] coordinates {(4,1)};
      \addplot[fill=gray] coordinates {(1,2)};
      \addplot[fill=gray] coordinates {(7,3)};
      \addplot[fill=gray] coordinates {(1,4)};
      \addplot[fill=gray] coordinates {(7,5)};
      \addplot[fill=gray] coordinates {(11,6)};
      \addplot[fill=gray] coordinates {(4,7)};
      \addplot[fill=gray] coordinates {(3,8)};
      \addplot[fill=dgreen] coordinates {(8,9)};
      \addplot[fill=black] coordinates {(7,10)};
      \addplot[fill=black] coordinates {(3,11)};
      \addplot[fill=black] coordinates {(4,12)};
      \addplot[fill=black] coordinates {(4,13)};
      \addplot[fill=black] coordinates {(8,14)};
      \addplot[fill=black] coordinates {(6,15)};
    \end{axis}
  \end{tikzpicture}
\end{frame}

\begin{frame}
  \subsection{Konservative vektorfelt}\label{subsec:Konservative-vektorfelt}
  \frametitle{Konservative vektorfelt}
  \includegraphics[scale=0.95]{../img/flytskjema-linjeintegral-2}
\end{frame}

\begin{frame}
  \frametitle{Konservative vektorfelt}
  \begin{definition}
    Et vektorfelt $\F \colon A \to \R^n$, hvor $A \subset \R^m$ er \emph{konservativt}
    dersom det eksisterer et skalarfelt $f$ med kontinuerlige partiellderiverte på
    $A$ slik at $\nabla f = (\frac{\partial f_1}{\partial x}, \frac{\partial
      f_2}{\partial y}) = (F_1, F_2) = \F$.
  \end{definition}
  \begin{theorem}
    \begin{itemize}
      \item $\F$ konservativt $\Longrightarrow$ $\curl \F = 0$
        \item Dersom følgende krav holder
        \begin{itemize}
          \item $\F$ er definert på et enkeltsammenhengende området $D$
          \item $\F$ har kontinuerlige partiellderiverte på hele $D$
          \item $\curl \F = 0$
        \end{itemize}
        Så har vi $\curl \F = 0$ $\Longrightarrow$ $\F$ konservativt.
    \end{itemize}
  \end{theorem}
\end{frame}

\begin{frame}
  \begin{oppgave}{V2015, Oppgave 8}
    La $\F \colon \R^3 \mapsto \R^3$ være gitt ved $\F(x, y, z) = yz\I +
    xz\J + xy\K$. Bestem
    %
    \begin{equation*}
      \int_C \F \cdot \dr\,,
    \end{equation*}
    %
    når $C$ har parametriseringen $\rr(t) = (\cos t)\I + (\sin t)\J + t\K$, $0 \leq t < \pi/4$.
  \end{oppgave}
\only<1-4>{$\F$ er definert på hele $\R^3$ (enkeltsammenhengende) og har kontinuerlige partiellderiverte. Dersom
$\curl \F = \vek{0}$ så er $\F$ et konservativ felt.}%
\only<5>{Siden $\F$ er et konservativt felt eksisterer det et skalarfelt
  (potensial) $f$ slik at $\nabla f = (\frac{\partial f}{\partial x},
  \frac{\partial f}{\partial y}, \frac{\partial f}{\partial z}) = (yz, xz, xy) =
  \F$.}%
    \only<6->{Siden $f = xyz$, $\nabla f = \F$ og $\F$ er konservativt bruker vi
      formelen
      \begin{equation*}
        \only<6>{\int_C \nabla f \cdot \dr = f(B) - f(A)}
        \only<7->{\int_C \F \cdot \dr}
        \only<7>{ = f\bigl(\rr(\pi/4)\bigr) - f\bigl(\rr(0)\bigr)}
        \only<8>{ = f\bigl(\cos \tfrac{\pi}{4}, \sin \tfrac{\pi}{4}, \tfrac{\pi}{4}\bigr)
                  - f\bigl(\cos 0, \sin 0, 0\bigr)}
          \only<9>{= f(\tfrac{1}{\sqrt{2}}, \tfrac{1}{\sqrt{2}}, \tfrac{\pi}{4}) - f(1,0,0)}
          \only<10->{= \frac{\pi}{8}}
      \end{equation*} hvor $B$ og $A$ er
      henholdsvis start og sluttpunktet til kurven $C$.}
\only<2->{%
  \begin{align*}
    \only<2-4>{%
  \hspace{-0.2cm}
      \curl \F
    & = \nabla \times \F =
  \begin{vmatrix}
    \I & \J & \K \\
    \frac{\partial}{\partial x} & \frac{\partial}{\partial y} & \frac{\partial}{\partial z}\\
    yz & xz & xy
  \end{vmatrix}}\only<3-4>{\\
    & = \I \left( \tfrac{\partial}{\partial y}(xy) - \tfrac{\partial}{\partial z}(xz) \right)
    - \J \left( \tfrac{\partial}{\partial x}(xy) - \tfrac{\partial}{\partial z}(yz) \right)
      + \K \left( \tfrac{\partial}{\partial x}(xz) - \tfrac{\partial}{\partial y}(yz) \right)\\}%
  \only<4>{
    & = \I \left( x - x \right)
    - \J \left( y - y \right)
      + \K \left( z - z\right)
      = \vek{0}}
  \only<5>{
      \diffp{f}{x} & = yz \Rightarrow f = xyz + C(y,z) \\
      \diffp{f}{y} & = xz \Rightarrow f = xyz + D(x,z) \\
      \diffp{f}{z} & = xy \Rightarrow f = xyz + E(x,y)
    }
  \end{align*}}
\only<5>{Velger $C(y,z)=D(x,z)=E(x,y)=0$ slik at $f(x,y,z) = xyz$.}
\end{frame}

%%% Local Variables:
%%% mode: latex
%%% TeX-master: "main"
%%% End:
 % Done 
% \begin{frame}
  \begin{tikzpicture}
    \begin{axis}[ xbar=0pt, /pgf/bar shift=0pt, legend style={ legend columns=4,
        at={(xticklabel cs:0.5)}, anchor=north, draw=none }, ytick={0,...,15},
      ytick style={draw=none},% <- added
      axis y line*=none, axis x line*=bottom, tick label
      style={font=\footnotesize}, legend style={font=\footnotesize}, label
      style={font=\footnotesize}, xtick style={draw=none},% <- added
      xtick={0,1,...,12}, width=.9\textwidth, bar width=3mm, y dir = reverse,
      xmin=0, xmax=13, area legend,
      y=5mm, enlarge y limits={abs=0.625},
      style={text=black}, every axis plot/.append style={fill},
      nodes near coords, nodes near coords,
      yticklabels={%
        {\topicref{Grenseverdier}},
        {\topicref{Gradient}},
        {\topicref{Epsilon-delta}},
        {\topicref{Kjerneregelen}},
        {\topicref{Taylor-approksimasjon}},
        {\topicref{Tangenter}},
        {\topicref{Kritiske-punkter}},
        {\topicref{Optimering}},
        {\topicref{Linjeintegral}},
        {\topicref{Konservative-vektorfelt}},
        {\topicref{Dobbelintegraler}},
        {\topicref{Integrasjonsrekkefolge}},
        {\topicref{Trippelintegral}},
        {\topicref{Greens-teorem}},
        {\topicref{Divergensteoremet}},
        {\topicref{Stokes-teorem}}}]
      \addplot[fill=gray] coordinates {(8,0)};
      \addplot[fill=gray] coordinates {(4,1)};
      \addplot[fill=gray] coordinates {(1,2)};
      \addplot[fill=gray] coordinates {(7,3)};
      \addplot[fill=gray] coordinates {(1,4)};
      \addplot[fill=gray] coordinates {(7,5)};
      \addplot[fill=gray] coordinates {(11,6)};
      \addplot[fill=gray] coordinates {(4,7)};
      \addplot[fill=gray] coordinates {(3,8)};
      \addplot[fill=gray] coordinates {(8,9)};
      \addplot[fill=dgreen] coordinates {(7,10)};
      \addplot[fill=black] coordinates {(3,11)};
      \addplot[fill=black] coordinates {(4,12)};
      \addplot[fill=black] coordinates {(4,13)};
      \addplot[fill=black] coordinates {(8,14)};
      \addplot[fill=black] coordinates {(6,15)};
    \end{axis}
  \end{tikzpicture}
\end{frame}

\begin{frame}
  \subsection{Dobbelintegraler}\label{subsec:Dobbelintegraler}
  \frametitle{Dobbelintegraler}
  \begin{enumerate}
    \item Integralet av en odde funksjon omkring et symmetrisk området er null
      \begin{equation*}
        \int_{-a}^{a} \only<1>{x}\only<2->{f(x)} \dx = 0
      \end{equation*}
      %
      \visible<2>{Hvor $f(-x) = -f(x)$.}
    \item Dersom integralet inneholder $x^2 + y^2$ bytt til polarkoordinater
      \begin{equation*}
        x^2 + y^2 = r^2 \quad x = r \cos \theta, y = r \sin \theta \quad \dx \dy = r \dd r \dT
      \end{equation*}

  \end{enumerate}
\end{frame}

\begin{frame}
  \begin{oppgave}{V2017, Oppgave 4} Beregn dobbelintegralet $\displaystyle\iint_{D}
    3 + x^3\dd(x,y)$ der $D$ er gitt ved $x^2 + y^2 \leq a^2$.
  \end{oppgave}
  \only<1->{\includegraphics[width=0.4\textwidth]{../img/dobbel-polar-xy}}
  \only<2->{\includegraphics[width=0.4\textwidth]{../img/dobbel-polar-zx}}
  \only<3>{\includegraphics[width=0.3\textwidth]{../img/dobbel-polar3d}}
  \only<4->{%
  \begin{equation*}
    \iint_{D} 3 + x^3\dd(x,y)
    \visible<5->{
      \only<6>{= 3 \iint_{D} \dd(x,y) + \iint_D x^3 \dd(x,y)}
      \only<7->{= 3 \underbrace{\iint_{D} \dd(x,y)}_{\pi a^2} + \underbrace{\iint_D x^3 \dd(x,y)}_{0}
    = 3 \pi a^2}}
  \end{equation*}}
\end{frame}

\begin{frame}
  \begin{oppgave}{V2017, Oppgave 4} Beregn dobbelintegralet $\displaystyle\iint_{D}
    3 + x^3\dd(x,y)$ der $D$ er gitt ved $x^2 + y^2 \leq a^2$.
  \end{oppgave}
  Bytter direkte til polarkoordinater
  %
  \begin{align*}
    \iint_D 3 + x^3 \dd(x,y)
    & = \int_0^{2\pi} \int_0^{a} (3 + (r \cos \theta)^3) r \dd r \dT \\
    & = \int_0^{2\pi} \biggl[ \only<1>{3r + r^4(\cos \theta) \dd
      r}\only<2->{\frac{3}{2}r^2 + \frac{r^5}{5} (\cos \theta)^3} \biggr]_0^a  \dT \visible<3->{\\
    & = \int_0^{2\pi} \frac{3}{2}a^2 + \frac{a^5}{5} (\cos \theta)^3 \dT \visible<4->{\\
    & = 3\pi a^2 + \frac{a^5}{5} \int_0^{2\pi} \only<1-4>{\textcolor{red}{(\cos \theta)^2}\cos \theta}\only<5>{\textcolor{red}{( 1 - (\sin \theta)^2)}\cos \theta}\only<6->{( 1 - (\sin \theta)^2) \cos \theta}\dT \visible<6->{\\
    & = 3\pi a^2 + \frac{a^5}{5} \left[ \only<6>{\int 1 - u^2 \du}\only<7>{u -
    \frac{1}{3}u^3 }\only<8->{\sin x - \frac{1}{3}(\sin x)^3} \right]_0^{2\pi}
    \only<9>{= 3\pi a^2}}}}
  \end{align*}
\end{frame}


%%% Local Variables:
%%% mode: latex
%%% TeX-master: "main"
%%% End:
        % Done 
% \begin{frame}
  \begin{tikzpicture}
    \begin{axis}[ xbar=0pt, /pgf/bar shift=0pt, legend style={ legend columns=4,
        at={(xticklabel cs:0.5)}, anchor=north, draw=none }, ytick={0,...,15},
      ytick style={draw=none},% <- added
      axis y line*=none, axis x line*=bottom, tick label
      style={font=\footnotesize}, legend style={font=\footnotesize}, label
      style={font=\footnotesize}, xtick style={draw=none},% <- added
      xtick={0,1,...,12}, width=.9\textwidth, bar width=3mm, y dir = reverse,
      xmin=0, xmax=13, area legend,
      y=5mm, enlarge y limits={abs=0.625},
      style={text=black}, every axis plot/.append style={fill},
      nodes near coords, nodes near coords,
      yticklabels={%
        {\topicref{Grenseverdier}},
        {\topicref{Gradient}},
        {\topicref{Epsilon-delta}},
        {\topicref{Kjerneregelen}},
        {\topicref{Taylor-approksimasjon}},
        {\topicref{Tangenter}},
        {\topicref{Kritiske-punkter}},
        {\topicref{Optimering}},
        {\topicref{Linjeintegral}},
        {\topicref{Konservative-vektorfelt}},
        {\topicref{Dobbelintegraler}},
        {\topicref{Integrasjonsrekkefolge}},
        {\topicref{Trippelintegral}},
        {\topicref{Greens-teorem}},
        {\topicref{Divergensteoremet}},
        {\topicref{Stokes-teorem}}}]
      \addplot[fill=black] coordinates {(8,0)};
      \addplot[fill=black] coordinates {(4,1)};
      \addplot[fill=black] coordinates {(1,2)};
      \addplot[fill=black] coordinates {(7,3)};
      \addplot[fill=black] coordinates {(1,4)};
      \addplot[fill=black] coordinates {(7,5)};
      \addplot[fill=black] coordinates {(11,6)};
      \addplot[fill=black] coordinates {(4,7)};
      \addplot[fill=black] coordinates {(3,8)};
      \addplot[fill=black] coordinates {(8,9)};
      \addplot[fill=black] coordinates {(7,10)};
      \addplot[fill=red] coordinates {(3,11)};
      \addplot[fill=black] coordinates {(4,12)};
      \addplot[fill=black] coordinates {(4,13)};
      \addplot[fill=black] coordinates {(8,14)};
      \addplot[fill=black] coordinates {(6,15)};
    \end{axis}
  \end{tikzpicture}
\end{frame}

\begin{frame}
  \subsection{Integrasjonsrekkefølge}\label{subsec:Integrasjonsrekkefolge}
  \begin{oppgave}{V2017, Oppgave 3}
    Skisser integrasjonsområdet for
    %
    $\displaystyle
      \int_0^1 \int_{\sqrt{x}}^1 f(x,y) \dy \dx,
    $
    %
    og bytt integrasjonsrekkefølgen til $\dx \dy$.
  \end{oppgave}
    \centerline{%
    \only<1>{\includegraphics[scale=0.8]{../img/dobbelIntegral1}}
    \only<2>{\includegraphics[scale=0.8]{../img/dobbelIntegral2}}
    \only<3>{\includegraphics[scale=0.8]{../img/dobbelIntegral3}}
    \only<4>{\includegraphics[scale=0.8]{../img/dobbelIntegral4}}
    \only<5>{\includegraphics[scale=0.8]{../img/dobbelIntegral5}}
    \only<6>{\includegraphics[scale=0.8]{../img/dobbelIntegral6}}
  }
  \vspace{-0.5cm}
  \begin{equation*}
      \int_{\only<1>{x = 0}\only<2>{\textcolor{blue}{x=0}}\only<3->{x=0}}^{\only<1-4>{x = 1}\only<5>{\textcolor{blue}{x=1}}\only<6->{x=1}} \int_{\only<1-2>{y = \sqrt{x}}\only<3>{\textcolor{blue}{y=\sqrt{x}}}\only<4->{y=\sqrt{x}}}^{\only<1-3>{y = 1}\only<4>{\textcolor{blue}{y=0}}\only<5->{y=0}} f(x,y) \dy \dx,
  \end{equation*}
\end{frame}

\begin{frame}
  \begin{oppgave}{V2017, Oppgave 3}
    Skisser integrasjonsområdet for
    %
    $\displaystyle
      \int_0^1 \int_{\sqrt{x}}^1 f(x,y) \dy \dx,
    $
    %
    og bytt integrasjonsrekkefølgen til $\dx \dy$.
  \end{oppgave}
  \centerline{%
    \only<1>{\includegraphics[scale=0.8]{../img/dobbelIntegral10}}
    \only<2>{\includegraphics[scale=0.8]{../img/dobbelIntegral11}}
    \only<3>{\includegraphics[scale=0.8]{../img/dobbelIntegral12}}
    \only<4>{\includegraphics[scale=0.8]{../img/dobbelIntegral13}}
    \only<5>{\includegraphics[scale=0.8]{../img/dobbelIntegral14}}
    \only<6-7>{\includegraphics[scale=0.8]{../img/dobbelIntegral15}}
  }
  \only<1-7>{
  \vspace{-0.5cm}
  \begin{equation*}
    \int_0^1 \int_{\sqrt{x}}^{1} f(x,y) \dy \dx
    =
    \int_{\only<1-4>{?}\only<5>{\textcolor{red}{y=0}}\only<6->{0}}^{\only<1-5>{?}\only<6>{\textcolor{red}{y=1}}\only<7>{1}} \int_{\only<1>{?}\only<2>{\textcolor{blue}{x=0}}\only<3->{0}}^{\only<1-2>{?}\only<3>{\textcolor{blue}{x=y^2}}\only<4->{y^2}} f(x,y) \dx \dy
  \end{equation*}}%
  \only<8>{
  Har at $0 \leq x \leq 1$ og $\sqrt{x} \leq y \leq 1$. Dette fører til at
  %
  \begin{equation*}
    \sqrt{0} \leq y \leq 1
  \end{equation*}
  %
  Ved å kvadrere $\sqrt{x} \leq y \leq 1$ får vi $x \leq y^2 \leq 1$. Videre vet
  vi at $0 \leq x$ slik at
  %
  \begin{equation*}
    0 \leq x \leq y^2
  \end{equation*}}
\end{frame}

%%% Local Variables:
%%% mode: latex
%%% TeX-master: "main"
%%% End:
  % Done  
% \begin{frame}
  \begin{tikzpicture}
    \begin{axis}[ xbar=0pt, /pgf/bar shift=0pt, legend style={ legend columns=4,
        at={(xticklabel cs:0.5)}, anchor=north, draw=none }, ytick={0,...,15},
      ytick style={draw=none},% <- added
      axis y line*=none, axis x line*=bottom, tick label
      style={font=\footnotesize}, legend style={font=\footnotesize}, label
      style={font=\footnotesize}, xtick style={draw=none},% <- added
      xtick={0,1,...,12}, width=.9\textwidth, bar width=3mm, y dir = reverse,
      xmin=0, xmax=13, area legend,
      y=5mm, enlarge y limits={abs=0.625},
      style={text=black}, every axis plot/.append style={fill},
      nodes near coords, nodes near coords,
      yticklabels={%
        {\topicref{Grenseverdier}},
        {\topicref{Gradient}},
        {\topicref{Epsilon-delta}},
        {\topicref{Kjerneregelen}},
        {\topicref{Taylor-approksimasjon}},
        {\topicref{Tangenter}},
        {\topicref{Kritiske-punkter}},
        {\topicref{Optimering}},
        {\topicref{Linjeintegral}},
        {\topicref{Konservative-vektorfelt}},
        {\topicref{Dobbelintegraler}},
        {\topicref{Integrasjonsrekkefolge}},
        {\topicref{Trippelintegral}},
        {\topicref{Greens-teorem}},
        {\topicref{Divergensteoremet}},
        {\topicref{Stokes-teorem}}}]
      \addplot[fill=black] coordinates {(8,0)};
      \addplot[fill=black] coordinates {(4,1)};
      \addplot[fill=black] coordinates {(1,2)};
      \addplot[fill=black] coordinates {(7,3)};
      \addplot[fill=black] coordinates {(1,4)};
      \addplot[fill=black] coordinates {(7,5)};
      \addplot[fill=black] coordinates {(11,6)};
      \addplot[fill=black] coordinates {(4,7)};
      \addplot[fill=black] coordinates {(3,8)};
      \addplot[fill=black] coordinates {(8,9)};
      \addplot[fill=black] coordinates {(7,10)};
      \addplot[fill=black] coordinates {(3,11)};
      \addplot[fill=red] coordinates {(4,12)};
      \addplot[fill=black] coordinates {(4,13)};
      \addplot[fill=black] coordinates {(8,14)};
      \addplot[fill=black] coordinates {(6,15)};
    \end{axis}
  \end{tikzpicture}
\end{frame}

\begin{frame}
  \subsection{Trippelintegral}\label{subsec:Trippelintegral}
  \frametitle{Trippelintegral}
  \begin{enumerate}
    \item Tegn integrasjonsområdet i $xy$, $xz$ og $yz$-planet
    \item Bestem grensene
    \item Gjør forenklinger med hensyn på symmetri
    \item Beregn trippelintegralet og bytt grenser om nødvendig
  \end{enumerate}
\end{frame}

\begin{frame}
    \begin{oppgave}{}
    Bestem volumet til $R = \{ (x,y,z)\in\R^3 \mid 0 \leq z \leq 1 - \abs{x} - \abs{y}\}$ når
    massetettheten er gitt som $f(x,y,z) = xy + z^2$.
  \end{oppgave}
\end{frame}

\begin{frame}\centerline{Viser området avgrenset av $0 \leq z \leq 1 - \abs{x} - \abs{y}$ for
  ulike plan.}
  \centerline{%
  \includegraphics[width = 0.45\textwidth]{../img/pyramid1}%
  \includegraphics[width = 0.45\textwidth]{../img/pyramid2}}

\centerline
\visible<2>{\includegraphics[width = 0.45\textwidth]{../img/pyramid}}}
\end{frame}

\begin{frame}
    \begin{oppgave}{}
    Bestem volumet til $R = \{ (x,y,z)\in\R^3 \mid 0 \leq z \leq 1 - \abs{x} - \abs{y}\}$ når
    massetettheten er gitt som $f(x,y,z) = xy + z^n$, $n \in \mathbb{Z}$.
  \end{oppgave}
  Siden $\int_{-a}^a f(x) \dx = 0$ når $f$ er en odde funksjon, og $x$ er odde så
er $\iiint_{R}xy \dV = 0$. Da området vi integrerer over er symmetrisk. Volumet
til den blå og røde trekanten er like store, men med motsatt fortegn.
\begin{align*}
  M & = \iiint_R f(x,y,z) \dV = \iiint_R z^n \dV\\
    \only<2>{& = \int_{x=?}^{x=?} \int_{y=f(x)}^{y=g(x)}\int_{z=u(x,y)}^{z=v(x,y)} z^n \dz \dy \dx \\}
    \only<3->{& = \int_{x=-1}^{x=1} \int_{y=-1+\abs{x}}^{y=1-\abs{x}}\int_{z=0}^{z=1-\abs{x}-\abs{y}} z^n \dz \dy \dx\\}
    \visible<4->{& = 4\int_{0}^{1} \int_{0}^{1-x}\int_{0}^{1-x-y} z^n \dz \dy \dx \\}
    \visible<5->{& = 4\int_{0}^{1} \int_{0}^{1-x} \frac{(1-x-y)^{n+1}}{n+1} \dy \dx \\
    & = 4\int_{0}^{1} \frac{(1-x)^{n+2}}{(n+1)(n+2)}  \dx
      = \frac{4}{(n+1)(n+2)(n+3)}}
\end{align*}
\end{frame}




%%% Local Variables:
%%% mode: latex
%%% TeX-master: "main"
%%% End:

% \begin{frame}
  \begin{tikzpicture}
    \begin{axis}[ xbar=0pt, /pgf/bar shift=0pt, legend style={ legend columns=4,
        at={(xticklabel cs:0.5)}, anchor=north, draw=none }, ytick={0,...,15},
      ytick style={draw=none},% <- added
      axis y line*=none, axis x line*=bottom, tick label
      style={font=\footnotesize}, legend style={font=\footnotesize}, label
      style={font=\footnotesize}, xtick style={draw=none},% <- added
      xtick={0,1,...,12}, width=.9\textwidth, bar width=3mm, y dir = reverse,
      xmin=0, xmax=13, area legend,
      y=5mm, enlarge y limits={abs=0.625},
      style={text=black}, every axis plot/.append style={fill},
      nodes near coords, nodes near coords,
      yticklabels={%
        {\topicref{Grenseverdier}},
        {\topicref{Gradient}},
        {\topicref{Epsilon-delta}},
        {\topicref{Kjerneregelen}},
        {\topicref{Taylor-approksimasjon}},
        {\topicref{Tangenter}},
        {\topicref{Kritiske-punkter}},
        {\topicref{Optimering}},
        {\topicref{Linjeintegral}},
        {\topicref{Konservative-vektorfelt}},
        {\topicref{Dobbelintegraler}},
        {\topicref{Integrasjonsrekkefolge}},
        {\topicref{Trippelintegral}},
        {\topicref{Greens-teorem}},
        {\topicref{Divergensteoremet}},
        {\topicref{Stokes-teorem}}}]
      \addplot[fill=gray] coordinates {(8,0)};
      \addplot[fill=gray] coordinates {(4,1)};
      \addplot[fill=gray] coordinates {(1,2)};
      \addplot[fill=gray] coordinates {(7,3)};
      \addplot[fill=gray] coordinates {(1,4)};
      \addplot[fill=gray] coordinates {(7,5)};
      \addplot[fill=gray] coordinates {(11,6)};
      \addplot[fill=gray] coordinates {(4,7)};
      \addplot[fill=gray] coordinates {(3,8)};
      \addplot[fill=gray] coordinates {(8,9)};
      \addplot[fill=gray] coordinates {(7,10)};
      \addplot[fill=gray] coordinates {(3,11)};
      \addplot[fill=gray] coordinates {(4,12)};
      \addplot[fill=dgreen] coordinates {(4,13)};
      \addplot[fill=black] coordinates {(8,14)};
      \addplot[fill=black] coordinates {(6,15)};
    \end{axis}
  \end{tikzpicture}
\end{frame}

\begin{frame}
  \centerline{%
    \only<1>{\includegraphics[scale=0.95]{../img/flytskjema-linjeintegral-4}}%
  }
  $\partial D$ er en kurve slik at $\partial D \colon \rr(t), \ a \leq t \leq b$
  med $\rr(a)=A$ og $\rr(b) = B$. \\
  $D$ er området avgrenset av $\partial D$. Dersom $\F$ er konservativ er $\nabla f = \F$.
\end{frame}

\begin{frame}
    \subsection{Greens teorem}\label{subsec:Greens-teorem}
  \frametitle{Greens teorem}
  \begin{theorem}[Greens theorem]
    La $\partial D$ være en positivt orientert, stykkevis glatt, enkel, lukket
    kurve og la $D$ være området innelukket av $\partial D$. Dersom $\F(x,y) = F_1(x,y)\I +
    F_2(x,y)\J$ og $\F$ har kontinuerlige partiellderiverte på hele $D$ så er
    %
    \begin{equation*}
      \int_{\partial D} \F \cdot \dr
      = \int_{\partial D} F_1 \dx + F_2 \dy
      = \iint_D \left( \diffp{{F_2}}{x} - \diffp{{F_1}}{y} \right) \dd (x,y).
    \end{equation*}
  \end{theorem}
  \begin{figure}
  \centering
  \begin{minipage}{.45\textwidth}
    \centering
  \includegraphics[width=\linewidth]{../img/greens-1.png}
\end{minipage}
\begin{minipage}{.45\textwidth}
    \centering
  \includegraphics[width=\linewidth]{../img/greens-2.png}
\end{minipage}
\end{figure}

\end{frame}

\begin{frame}
  \frametitle{Greens teorem}
  \begin{intuisjon}
    \centering
    \begin{minipage}{.30\textwidth}
      \centering
      \includegraphics[width=\linewidth]{../img/Greens-intuition-1}%
    \end{minipage}\qquad
    \begin{minipage}{.30\textwidth}
      \centering
      \only<1>{\phantom{11111111111111111111}}
      \only<2>{\includegraphics[width=0.95\linewidth]{../img/Greens-intuition-2}}
      \only<3->{\includegraphics[width=0.95\linewidth]{../img/Greens-intuition-3}}
    \end{minipage}
    \begin{equation*}
      \only<1>{\hspace{1cm}}
      \only<4-5>{\hspace{0.25cm}}
      \only<5>{\hspace{0.5cm}}
      \int_{\partial D} \F \cdot \dr \visible<2->{ \only<2-3>{\hspace{0.5cm}}\only<4-5>{\hspace{0.25cm}}\quad = \quad \only<4-5>{\hspace{0.25cm}}\only<2-3>{\hspace{0.5cm}}\only<5>{\iint_{D} \left( \diffp{{F_2}}{x} - \diffp{{F_1}}{y} \right) \dd (x,y)}\only<4>{\iint_{D} \2d-curl \F \dd (x,y)}\only<2-3>{\int_{\partial D} \F \cdot \dr}}
      \only<1>{\phantom{\quad = \quad \iint_{D} \2d-curl \F \dd (x,y)}}
    \end{equation*}
\only<5>{
    Med andre ord kanseleres alle de mikroskopiske sirkulasjonene slik at en
  står igjen med den makroskopiske sirkulasjonen, altså linjeintegralet.}
  \end{intuisjon}
\end{frame}

\begin{frame}
  \begin{oppgave}{V2014, Oppgave , Oppgave 5} Regn ut
    $\displaystyle\int_{\partial D} \bigl(  5y - e^{\sin x} \bigr) \dx + \bigl(
    10x - \sin(y^3 + 8y) \bigr) \dy $ der $\partial D$ er en sirkel med radus $2$ og
    sentrum $(a,b)$. Spesifiser om du regner integralet med eller mot klokka.
  \end{oppgave}
  \only<1-3>{\pause
  \textbf{Steg 1:} Siden linjeintegralet vårt er på formen $\displaystyle\int_C P \dx + Q \dy$
  og vi integrerer over en lukket kurve er det logisk å tenke Greens.

  \pause
  \textbf{Steg 2:} Sjekker om vilkårene for å bruke Greens er oppfylt:
  \begin{itemize}
    \item Kurven $\partial D$ er lukket, glatt og enkel.
    \item Da alle polynomer, eksponensialfunksjoner og sinus/cosinus er deriverbare, har $\F$ kontinuerlige
      partiellderiverte.
  \end{itemize}
  \textbf{Merk:} For at for å bruke Green's må $\partial D$ orienteres
  \emph{mot-klokken}.}

  \visible<4->{
  \textbf{Steg 3:} Bruk Green's til å beregn linjeintegralet
  \begin{align*}
    I = & \iint_{\partial D} \bigl(  5y - e^{\sin x} \bigr) \dx + \bigl(
          10x - \sin(y^3 + 8y) \bigr) \dy \\
    \overset{\text{Greens teorem}}{=} & \iint_{D} \left( \diffp{}{x}\left( 10x - \sin(y^3 + 8y) \right) - \diffp{}{y}\left( 5y - e^{\sin x} \right)\right) \dd(x,y) \\
    \only<5->{= & \iint_{D} \left( \diffp{}{x}\left( 10x - 0 \right) - \diffp{}{y}\left( 5y - 0 \right)\right) \dd(x,y)} \\
    \only<6>{= & \iint_{D} ( 10 - 5) \dd(x,y)\\}
    \only<7->{=& 5 \iint_{D} \dd(x,y) }
    \only<8->{ = 5 \text{Areal}(D) }
    \only<9->{  = 5 (\pi \cdot 2^2) = 20 \pi}
  \end{align*}}
\end{frame}


%%% Local Variables:
%%% mode: latex
%%% TeX-master: "main"
%%% End:
           % Done
% \begin{frame}
  \subsection{Overflateintegral}
  \frametitle{Overflateintegral}
  \centerline{%
    \only<1>{\includegraphics{../img/flytskjema-overflateintegral}}
    \only<2>{\includegraphics{../img/flytskjema-overflateintegral-0}}
  }
  $S$ er en parametrisert flate $S \colon \rr(u,v), (u,v) \in R \subseteq \R^2$ med rand $\partial S$. \\
  $T$ er volumet med overflate $S$. $B$ er en tiltenkt flate (lokk til $S$).
\end{frame}

\begin{frame}
  \begin{enumerate}
    \item Tegn integrasjonsområdet og bestem integrasjonsgrensene
    \item Bestem parametriseringen $\rr(u,v) \colon (u,v) \in R$ og
      $
        \diffp{\rr}{u}\times\diffp{\rr}{v}
      $.
    \item Dersom $\F(x,y)$ er et vektorfelt blir fluksen ut av oveflaten
      \begin{align*}
        \iint_S \F \cdot \dS & =
        \iint_R \F(\rr(u,v)) \cdot \left( \diffp{\rr}{u}\times\diffp{\rr}{v} \right)\dd(x,y)\,.
      \intertext{Dersom $z = g(x,y)$ så er $\rr(x,y) = (x,y,g(x,y))$, $(x,y) \in R$ og}
        \iint_S \F \cdot \dS & =
        \iint_R\,(F_1, F_2, F_3) \cdot (-g_x,-g_y,1) \dd(x,y).
      \end{align*}
    \item Dersom $g(x,y)$ er et skalarfelt er overflateintegralet gitt som
      \begin{align*}
        A & =
        \iint_S\,\norm*{\diffp{\rr}{u}\times\diffp{\rr}{v}}\dd(u,v)\,,
      \intertext{I spesialtilfelle $z = g(x,y)$ er $\rr(u,v) = (x,y,g(x,y))$ og}
      %
        A & =
        \iint_T \sqrt{\left( \diffp{f}{x} \right)^2+\left( \diffp{f}{y} \right)^2+1} \dd(x,y).
      \end{align*}
  \end{enumerate}
\end{frame}

\begin{frame}
  \begin{oppgave}{V2015, Oppgave 5b}
    La $g(x,y) = 1+\frac{y^2}{2x}$ være definert på $g\colon D \to \R$ hvor
    $D = \{(x,y) \in \R^2 \mid 1 \leq x \leq 9, -3 \sqrt{x} \leq y \leq 3 \sqrt{x}\}$
    \\ bestem arealet av flaten $S = \{(x,y,z)\in \R^3 \mid z = g(z,y), (x,y) \in D\}$.
  \end{oppgave}
  \begin{enumerate}
    \item Tegn integrasjonsområdet og bestem integrasjonsgrensene
  \end{enumerate}
  \begin{minipage}[b]{0.45\textwidth}\centering
    \includegraphics{../img/overflateintegral-2015V}
  \end{minipage}\hfill
  \begin{minipage}[b]{0.45\textwidth}\centering\vspace*{-3cm}
  \begin{align*}
    \begin{split}
    1 \leq & x \leq 9 \\
    -3 \sqrt{x} \leq & y \leq 3 \sqrt{x}
  \end{split} \\ \ & \Downarrow \ \\
   \iint_D \dd(x,y) & = \int_1^{9} \int_{-3\sqrt{x}}^{3\sqrt{x}}\dy \dx
  \end{align*}
  \vspace{1cm}
  \phantom{a//a//a//a//a//a//a//a//a//a//a}
  \end{minipage}
\end{frame}

\begin{frame}
  \begin{oppgave}{V2015, Oppgave 5b}
    La $g(x,y) = 1+\frac{y^2}{2x}$ være definert på $g\colon D \to \R$ hvor
    $D = \{(x,y) \in \R^2 \mid 1 \leq x \leq 9, -3 \sqrt{x} \leq y \leq 3 \sqrt{x}\}$
    bestem arealet av flaten $S = \{(x,y,z)\in \R^3 \mid z = g(z,y), (x,y) \in D\}$.
  \end{oppgave}
  \begin{enumerate}
      \stepcounter{enumi}
    \item Bestem parametriseringen $\rr(u,v) \colon (u,v) \in R$ og
      $
      \diffp{\rr}{u}\times\diffp{\rr}{v}
      $.
  \end{enumerate}
      Parametriseringen vil være: $\mathbf{r}(x, y)=(x, y, g(x,y))$, $(x,y)\in D$  Slik at $\frac{\partial
        \mathbf{r}}{\partial x}=(1, 0, g_x(x,y))$, og $\frac{\partial
        \mathbf{r}}{\partial y}=(0, 1, g_y(x,y))$.  Så,
      \begin{equation}
        \diffp{\rr}{x}\times\diffp{\rr}{y}
        = \begin{vmatrix}
          \I & \J & \K \\
          1 & 0 & \diffp{g}{x} \\[0.2cm]
          0 & 1 & \diffp{g}{y}
        \end{vmatrix}
        = \left( -\diffp{g}{x}, -\diffp{g}{y}, 1 \right)
      \end{equation}
  \end{frame}

\begin{frame}
  \begin{oppgave}{V2015, Oppgave 5b}
    La $g(x,y) = 1+\frac{y^2}{2x}$ være definert på $g\colon D \to \R$ hvor
    $D = \{(x,y) \in \R^2 \mid 1 \leq x \leq 9, -3 \sqrt{x} \leq y \leq 3 \sqrt{x}\}$
    bestem arealet av flaten $S = \{(x,y,z)\in \R^3 \mid z = g(z,y), (x,y) \in D\}$.
  \end{oppgave}
  \begin{enumerate}
      \stepcounter{enumi}
      \stepcounter{enumi}
    \item Dersom $g(x,y)$ er et skalarfelt er overflateintegralet gitt som
  \end{enumerate}
      \begin{align*}
        A = \iint\limits_S \dSS
        & = \iint\limits_D \only<1-2>{\norm*{  \only<1>{\diffp{\rr}{x}\times\diffp{\rr}{y}}  \only<2>{\left(-\diffp{g}{y}, -\diffp{g}{y}, 1\right)}}} \only<3->{\sqrt{1 + \left( \diffp{g}{x} \right)^2+\left( \diffp{g}{y} \right)^2}} \dd (x,y) \only<4->{\\
        & = \int_1^9 \int_{-3\sqrt{x}}^{3\sqrt{x}}\sqrt{\only<4>{1 + \left( -\frac{y^2}{2x^2} \right)^2 + \left( \frac{y}{x} \right)^2}\only<5>{\textcolor{red}{1}^2 + 2 \cdot \textcolor{red}{1} \cdot \textcolor{blue}{  \frac{y^2}{2x^2}} + \left( \textcolor{blue}{\frac{y^2}{2x^2}} \right)^2}\only<6>{\left( \textcolor{red}{1} + \textcolor{blue}{\frac{y^2}{2x^2}} \right)^2}\only<7->{\left( 1 + \frac{y^2}{2x^2} \right)^2}}\dy\dx}
          \only<5>{\\
          [\textcolor{red}{a}^2 + 2\cdot\textcolor{red}{a}\cdot\textcolor{blue}{b} + \textcolor{blue}{b}^2 & = (\textcolor{red}{a} + \textcolor{blue}{b})^2]}
                                                                                                             \only<6-7>{\\& = \int_1^9 \only<6>{\left[ y + \frac{y^3}{6x^2} \right]_{-3\sqrt{x}}^{3\sqrt{x}}}\only<7>{6 \sqrt{x} + \frac{9}{\sqrt{x}}}\dx}
                                                                                                                          \only<8>{\\& = \left[ 4x^{3/2} + 18x^{1/2} \right]_1^9}
                                                                                                                          \only<9>{\\& = 140}
      \end{align*}
    \only<3>{%
    \begin{intuisjon}
      Fra Analyse I husker vi at lengden av en funksjon var gitt som
      %
      \begin{equation*}
        L = \int_a^b \sqrt{1 + \left( \diffp{f}{x} \right)^2} \dx
      \end{equation*}
      %
      Altså ser vi at et overflateintegral kan betraktes som et to-dimensjonalt
      buelengdeintegral dersom $z = g(x,y)$.
    \end{intuisjon}}
\end{frame}


%%% Local Variables:
%%% mode: latex
%%% TeX-master: "main"
%%% End:
       % Done
% \begin{frame}
  \begin{tikzpicture}
    \begin{axis}[ xbar=0pt, /pgf/bar shift=0pt, legend style={ legend columns=4,
        at={(xticklabel cs:0.5)}, anchor=north, draw=none }, ytick={0,...,15},
      ytick style={draw=none},% <- added
      axis y line*=none, axis x line*=bottom, tick label
      style={font=\footnotesize}, legend style={font=\footnotesize}, label
      style={font=\footnotesize}, xtick style={draw=none},% <- added
      xtick={0,1,...,12}, width=.9\textwidth, bar width=3mm, y dir = reverse,
      xmin=0, xmax=13, area legend,
      y=5mm, enlarge y limits={abs=0.625},
      style={text=black}, every axis plot/.append style={fill},
      nodes near coords, nodes near coords,
      yticklabels={%
        {\topicref{Grenseverdier}},
        {\topicref{Gradient}},
        {\topicref{Epsilon-delta}},
        {\topicref{Kjerneregelen}},
        {\topicref{Taylor-approksimasjon}},
        {\topicref{Tangenter}},
        {\topicref{Kritiske-punkter}},
        {\topicref{Optimering}},
        {\topicref{Linjeintegral}},
        {\topicref{Konservative-vektorfelt}},
        {\topicref{Dobbelintegraler}},
        {\topicref{Integrasjonsrekkefolge}},
        {\topicref{Trippelintegral}},
        {\topicref{Greens-teorem}},
        {\topicref{Divergensteoremet}},
        {\topicref{Stokes-teorem}}}]
      \addplot[fill=gray] coordinates {(8,0)};
      \addplot[fill=gray] coordinates {(4,1)};
      \addplot[fill=gray] coordinates {(1,2)};
      \addplot[fill=gray] coordinates {(7,3)};
      \addplot[fill=gray] coordinates {(1,4)};
      \addplot[fill=gray] coordinates {(7,5)};
      \addplot[fill=gray] coordinates {(11,6)};
      \addplot[fill=gray] coordinates {(4,7)};
      \addplot[fill=gray] coordinates {(3,8)};
      \addplot[fill=gray] coordinates {(8,9)};
      \addplot[fill=gray] coordinates {(7,10)};
      \addplot[fill=gray] coordinates {(3,11)};
      \addplot[fill=gray] coordinates {(4,12)};
      \addplot[fill=gray] coordinates {(4,13)};
      \addplot[fill=dgreen] coordinates {(8,14)};
      \addplot[fill=black] coordinates {(6,15)};
    \end{axis}
  \end{tikzpicture}
\end{frame}

\begin{frame}
  \subsection{Divergensteoremet}\label{subsec:Divergensteoremet}
  \frametitle{Divergensteoremet}
    \centerline{%
    \includegraphics{../img/flytskjema-overflateintegral-1}
  }
\end{frame}

\begin{frame}
  \begin{theorem}[Divergensteoremet]
    La $V$ være et lukket volum i rommet som har en parametriserbar overflate
    $S$, og la enhetsnormalen $\vek{n}$ peke ut av $V$. Dersom $\F$ er et
    vektorfelt som har kontinuerlige partiellderiverte på $V$ da er
    %
    \begin{equation*}
      \iint_S \F \cdot \vek{n} \dS = \iiint_V \div \F \dV
    \end{equation*}
  \end{theorem}
  \begin{minipage}{.45\textwidth}
    \begin{intuisjon}
      %
      Anta du har en \emph{full} bolle med vann, og heller på mer vann. Hva
      skjer? Jo det renner vann ut av overflaten til bollen. Med andre ord er
      summen av endring i vesketetthet ($\iiint_V \div \F \dV$) lik
      hvor mye væske som forlater overflaten ($\iint_S \F \cdot \dSS$).
      %
    \end{intuisjon}
\end{minipage}
\begin{minipage}{.45\textwidth}
  \centering
    \includegraphics[scale=0.75]{../img/bolle2.png}
  \end{minipage}
\end{frame}

\begin{frame}
  \begin{oppgave}{V2012, Oppgave 5} Finn fluksen av vektorfeltet $F(x, y, z) =
    (\sin y) 2 \I + x\J + \only<1-5>{x}\only<6>{3}z\K$ inn i enhetskula med sentrum i origo.
  \end{oppgave}
  \visible<2->{
  \textbf{To krav:}
  \begin{enumerate}
      \item Er flaten $V$ avgrenset av randen $S$ lukket og
      enkeltsammenhengende?
      Ja, enhetskula er åpenbart lukket og enkeltsammenhengdende.
      \item Har vektor feltet $F$ kontinuerlige partiellderiverte definert på
      $W$. Ja, alle polynomer og $\sin x$ er deriverbare.
  \end{enumerate}}\only<3->{
  Har at normalvektoren er $-\vek{n}$ siden den skulle peke \emph{inn} i kula.
  \begin{align*}
    \iint_{S} \F \cdot (-\vek{n}) \dS
    & = -\iint_{S} \F \cdot \vek{n}\dS
    \overset{\text{Gauss}}{=} -\iiint_{V} \div \F \dV \\
    & = -\iiint_V 0 + 0 + \only<1-5>{x}\only<6>{3} \dV
      = -\iiint_V \only<1-5>{x}\only<6>{3} \dV\only<5>{=0}\only<6>{=\,?}
  \end{align*}}
\end{frame}

\begin{frame}
  \frametitle{Divergensteoremet}
    \centerline{%
    \includegraphics{../img/flytskjema-overflateintegral-4}
  }
\end{frame}


\begin{frame}
    \begin{example}
        Halvkulen $T$ er bestemt av
        %
        $x^2 + y^2 + z^2 \leq 1$ og $z \geq 0$
        %
        Vi lar $S$ betegne den krumme delen av overflaten til $T$.
        Vektorfeltet $\mathbf{F}$ er gitt ved
        %
        \begin{equation*}
            \mathbf{F}(x,y,z)
            = x(z+z^2)\I - (yz^2 + xe^y)\J + (xze^y + 1)\K
        \end{equation*}
        %
        Bestem fluksintegralet
        %
        $ \displaystyle
        \iint_{S} \mathbf{F} \cdot \vek{n} \, \mathrm{d}S
        $
        %
        der $\vek{n}$ har positiv $z$-komponent.
      \end{example}
      \begin{enumerate}
        \item Divergens: $ \only<2->{\text{div } \mathbf{F}=(z + z^2)-(z^2 + xe^y)
          + xe^y} \only<3->{=z}$
      \end{enumerate}
      \only<1-2>{
      \begin{equation*}
        \div \F
        = \diffp{}{x}\bigl( x(z+z^2) \bigr) - \diffp{}{y}\bigl( yz^2 + xe^y \bigr) + \diffp{}{z}\bigl( xze^y + 1 \bigr)
      \end{equation*}}
      \begin{enumerate}\setcounter{enumi}{1}
        \item Bruke divergensteoremet på \emph{hele} overflaten til $T$
      \end{enumerate}
      \only<3->{
        \begin{equation*}
          \iint_{T} \F \cdot \vek \dS
          \overset{\text{Gauss}}{=} \iiint_V \only<3>{\div \F}\only<4->{z} \dV
          \only<5>{= \int_0^{\pi/2}\int_0^{2\pi}\int_0^1 \rho \cos \varphi (\rho^2
          \sin \varphi )\dd \rho \dd \varphi \dT}
          \only<6>{= 2\pi \left[ \frac{\rho^4}{4} \right]_0^1 \left[ -\frac{(\cos \varphi)^2}{2} \right]_0^{\pi/2}}
          \only<7>{= 2\pi \left[  \frac{1}{4} - 0 \right]\left[  \frac{1^2}{2} - 0\right]= \frac{\pi}{4}}
        \end{equation*}
      }
      \begin{enumerate}\setcounter{enumi}{2}
        \item Finne fluksen ut av $S$.
    \end{enumerate}

\end{frame}

\begin{frame}
%\begin{frame}{Pixelweise Segmentierung}
    \begin{figure}[ht]
        \begin{minipage}[b]{0.30\linewidth}
            \centering
            \includegraphics[width=\textwidth]{../img/3-Lukket.pdf}
%            \caption*{$\iint_{D} F \cdot \hat{N} \mathrm{d}S$}
        \end{minipage}
        \hspace{0.30cm}
        \begin{minipage}[b]{0.30\linewidth}
            \centering
            \includegraphics[width=\textwidth]{../img/1-Topp.pdf}
%            \caption*{$\iint_{S} F \cdot \hat{N} \mathrm{d}S$}
        \end{minipage}
        \hspace{0.30cm}
        \begin{minipage}[b]{0.30\linewidth}
            \centering
            \includegraphics[width=\textwidth]{../img/2-Bunn.pdf}
%            \caption*{$\iint_{B} F \cdot \hat{N} \mathrm{d}S$}
        \end{minipage}
    \end{figure}
\begin{align*}
    \iint_{T} \F \cdot \vek{n} \dS
    \qquad =  \qquad
    \iint_{S} \F \cdot \vek{n} \dS
    \qquad + \qquad
    \iint_{B} \F \cdot \vek{n} \dS
\end{align*}
\end{frame}

\begin{frame}
%\begin{frame}{Pixelweise Segmentierung}
    \begin{figure}[ht]
        \begin{minipage}[b]{0.30\linewidth}
            \centering
            \includegraphics[width=\textwidth]{../img/3-Lukket.pdf}
%            \caption*{$\iint_{D} F \cdot \hat{N} \mathrm{d}S$}
        \end{minipage}
        \hspace{0.30cm}
        \begin{minipage}[b]{0.30\linewidth}
            \centering
            \includegraphics[width=\textwidth]{../img/1-Topp.pdf}
%            \caption*{$\iint_{S} F \cdot \hat{N} \mathrm{d}S$}
        \end{minipage}
        \hspace{0.30cm}
        \begin{minipage}[b]{0.30\linewidth}
            \centering
            \includegraphics[width=\textwidth]{../img/2-Bunn.pdf}
%            \caption*{$\iint_{B} F \cdot \hat{N} \mathrm{d}S$}
        \end{minipage}
    \end{figure}
\begin{align*}
    \iiint_{T} \text{div }\F \dV
    \qquad =  \qquad
    \iint_{S} \F \cdot \vek{n} \dS
    \qquad + \qquad
    \iint_{B} \F \cdot \vek{n} \dS
\end{align*}
\end{frame}

\begin{frame}
%\begin{frame}{Pixelweise Segmentierung}
    \begin{figure}[ht]
        \begin{minipage}[b]{0.30\linewidth}
            \centering
            \includegraphics[width=\textwidth]{../img/1-Topp.pdf}
%            \caption*{$\iint_{D} F \cdot \hat{N} \mathrm{d}S$}
        \end{minipage}
        \hspace{0.30cm}
        \begin{minipage}[b]{0.30\linewidth}
            \centering
            \includegraphics[width=\textwidth]{../img/3-Lukket.pdf}
%            \caption*{$\iint_{S} F \cdot \hat{N} \mathrm{d}S$}
        \end{minipage}
        \hspace{0.30cm}
        \begin{minipage}[b]{0.30\linewidth}
            \centering
            \includegraphics[width=\textwidth]{../img/2-Bunn.pdf}
%            \caption*{$\iint_{B} F \cdot \hat{N} \mathrm{d}S$}
        \end{minipage}
    \end{figure}
\begin{align*}
    \iint_{S} \F \cdot \vek{n} \dS
    \qquad =  \qquad
    \iiint_{T} \div \F \dV
    \qquad - \qquad
    \iint_{B} \F \cdot \vek{n} \dS
\end{align*}
\end{frame}


\begin{frame}
  Reminder:
  $
    \mathbf{F}(x,y,z)
    = x(z+z^2)\I - (yz^2 + xe^y)\J + (xze^y + 1)\K
    $ \\ \medskip

    Beregner overflateintegralet \emph{ut} av bunnen. Siden normalvektoren må peke ut av flaten får vi $\vek{n} = (0,0,-1)$ Så
    %
    \begin{align*}
        \iint_{B} \F \cdot \vek{n} \dS
        & =  \iint_{B} \F \cdot (0,0-1)\dS  \\
        & = -\iint_{B} (x \cdot \only<2->{\textcolor{red}{0}}\only<1>{\textcolor{red}{z}} \cdot e^{y} + 1) \dS \visible<2->{
        = -\iint_{B} 1 \dS
        = - \pi}
    \end{align*}\visible<3->{
    Siden $z = 0$ og $B$ er en sirkel med radius $1$. Oppsumert har vi altså
\begin{align*}
    \iint_{S} \F \cdot \vek{n} \dS
    & =
    \iiint_{T} \div \F \dV
    -
    \iint_{B} \F \cdot \vek{n} \dS \\
    & = \frac{\pi}{4} \qquad \qquad \ \: \ - \qquad(-\pi) \\
    & = \frac{5\pi}{4}
\end{align*}}
\end{frame}

%%% Local Variables:
%%% mode: latex
%%% TeX-master: "main"
%%% End:

% \begin{frame}
  \begin{tikzpicture}
    \begin{axis}[ xbar=0pt, /pgf/bar shift=0pt, legend style={ legend columns=4,
        at={(xticklabel cs:0.5)}, anchor=north, draw=none }, ytick={0,...,15},
      ytick style={draw=none},% <- added
      axis y line*=none, axis x line*=bottom, tick label
      style={font=\footnotesize}, legend style={font=\footnotesize}, label
      style={font=\footnotesize}, xtick style={draw=none},% <- added
      xtick={0,1,...,12}, width=.9\textwidth, bar width=3mm, y dir = reverse,
      xmin=0, xmax=13, area legend,
      y=5mm, enlarge y limits={abs=0.625},
      style={text=black}, every axis plot/.append style={fill},
      nodes near coords, nodes near coords,
      yticklabels={%
        {\topicref{Grenseverdier}},
        {\topicref{Gradient}},
        {\topicref{Epsilon-delta}},
        {\topicref{Kjerneregelen}},
        {\topicref{Taylor-approksimasjon}},
        {\topicref{Tangenter}},
        {\topicref{Kritiske-punkter}},
        {\topicref{Optimering}},
        {\topicref{Linjeintegral}},
        {\topicref{Konservative-vektorfelt}},
        {\topicref{Dobbelintegraler}},
        {\topicref{Integrasjonsrekkefolge}},
        {\topicref{Trippelintegral}},
        {\topicref{Greens-teorem}},
        {\topicref{Divergensteoremet}},
        {\topicref{Stokes-teorem}}}]
      \addplot[fill=black] coordinates {(8,0)};
      \addplot[fill=black] coordinates {(4,1)};
      \addplot[fill=black] coordinates {(1,2)};
      \addplot[fill=black] coordinates {(7,3)};
      \addplot[fill=black] coordinates {(1,4)};
      \addplot[fill=black] coordinates {(7,5)};
      \addplot[fill=black] coordinates {(11,6)};
      \addplot[fill=black] coordinates {(4,7)};
      \addplot[fill=black] coordinates {(3,8)};
      \addplot[fill=black] coordinates {(8,9)};
      \addplot[fill=black] coordinates {(7,10)};
      \addplot[fill=black] coordinates {(3,11)};
      \addplot[fill=black] coordinates {(4,12)};
      \addplot[fill=black] coordinates {(4,13)};
      \addplot[fill=black] coordinates {(8,14)};
      \addplot[fill=red] coordinates {(6,15)};
    \end{axis}
  \end{tikzpicture}
\end{frame}

\begin{frame}
  \subsection{Stokes-teorem}\label{subsec:Stokes-teorem}
  \frametitle{Stokes-teorem}
  \centerline{%
    \includegraphics{../img/flytskjema-overflateintegral-2}
  }

\end{frame}
\begin{frame}
\begin{theorem}[Stokes' teorem]
  La $S$ være en orienterbar glatt overflate som er begrenset av en enkel,
  lukket stykkevis-glatt randkurve $C$ med positiv orientasjon. Dersom $\F$ er
  et vektorfelt med kontinuerlige partiellderiverte på $S$ så er
  \begin{equation*}
    \iint_S \curl \F \cdot \dSS = \int_C \F \cdot \dr
  \end{equation*}
\end{theorem}
\centerline{
  \includegraphics[scale=0.5]{../img/stokes-surface.png}
}
\end{frame}
\begin{frame}
  \begin{oppgave}{K2012, Oppgave 6c}
    Legemet $T$ er avgrenset av paraboloiden $z = 4x^2 +4y^2$ og planet $z = 4$.
    La $S$ være den delen av overflaten til $T$ som ligger på paraboloiden $z =
    4x^2 + 4y^2$ , og la $\vek{n}$ være enhetsnormalen til $S$ som peker ut av
    legemet $T$ . Vektorfeltet $F$ er
$
  \F(x, y, z) = \frac{yz}{8\pi}\I - \frac{x}{2\pi} \J + \frac{z}{4}\K
$. Regn ut 
\begin{equation*}
  \iint_S \curl \F \cdot \vek{n} \dS.
\end{equation*}
\end{oppgave}

\only<1->{
Flatene $z = 4x^2 + 4z^2$ og $z = 4$ skjærer hverandre i sirkelen $x^2 + y^2 =
1$, $z = 4$. }\only<2->{La $C$ betegne sirkelen med radius $r = 1$ og $z = 4$, orientert
\emph{med} urviseren sett ovenfra. Bruker parametriseringen
\begin{equation*}
  \rr(\theta) = (\sin \theta) \I + (\cos \theta) \J + 4 \K 
\end{equation*}}
\only<7>{
  \begin{equation*}
  \iint_S \curl \F \cdot \vek{n} \dS = 
  \int_C \F \cdot \dr= \frac{1}{2}
\end{equation*}}
\visible<3-6>{
  \vspace{-0.5cm}
\begin{equation*}
  \int_C \F \cdot \dr
  \only<3>{= \int_0^{2\pi} (0, -\frac{\sin t}{2\pi}, 1)(\sin t, -\sin t, 0) \dt}
  \only<4->{= \frac{1}{2\pi}\int_0^{2\pi} (\sin t)^2 \dt}
  \visible<5->{= \frac{1}{2\pi}\int_0^{2\pi} \frac{1}{2} \dt}
  \visible<6>{= \frac{1}{2}}
\end{equation*}}

\end{frame}

%%% Local Variables:
%%% mode: latex
%%% TeX-master: "main"
%%% End:


\end{document}
