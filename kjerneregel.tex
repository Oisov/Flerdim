

\seksjon{Kjerneregelen}

\oppgave[V2016, Oppgave 2] Anta at en flue beveger seg langs en kurve $C$ i
$\R^3$ slik at posisjonen til flue er gitt ved $\vek{c}(t) = t \I +
\frac{2}{3}\sqrt{2} t^{3/2} \J + \frac{1}{2}t^2 \K$ ved tiden $t
\subset [0, \infty)$.
    
\begin{enumerate}
    \stepcounter{enumi}
  \item La $T(x,y,z) = x^2 + xz + y$ være temperaturen i punktet $(x,y,z) \in
    \R^3$. Finn $\diff{T}{t} \vek{c}(t)$ altså temperaturendringen flua vil
    oppleve ved tiden $t = 1$.
\end{enumerate}

\oppgave[V2016, Oppgave 6] Med definisjonen $\nabla^2 := \nabla \cdot \nabla =
\diffp[2]{}{x} + \diffp[2]{}{y}$, kalles en $C^2$-funksjon $f \colon \R^2 \to
\R$ som tilfredstiller ligningen
%
\begin{equation*}
  \nabla^2 f = 0
\end{equation*}
%
for \emph{harmonisk}. Ved at for et vilkårlig reelt tall $k$ er funksjonen
$f(x,y) = e^{kx}(\cos ky)$ harmonisk.


\oppgave[V2015, Oppgave 4] Vis at hvis akselerasjonen $\vek{a}(t)$ til en
punktmasse alltid er perpendikulær på hastigheten $\vek{v}(t)$, så er farten
$\norm{\vek{v}}$ konstant. (Hint: $\norm{\vek{v}}^2 = \vek{v} \cdot \vek{v}$.


\oppgave[K2014, Oppgave 2] Gitt $z = f(x, y)$ der $f$ er en deriverbar funksjon
og $x = u^2 - v^2$, $y = v^2 - u^2$. Vis at
%
\begin{equation*}
  u \diffp{z}{v} + v \diffp{z}{u} = 0\,.
\end{equation*}

\oppgave[V2014, Oppgave 1] Gitt $z = f(x,y)$ der $f$ er en deriverbar funksjon,
$x = u + v$ og $y = u - v$. Finn $\diffp{z}{u}$, $\diffp{z}{v}$ og vis at
%
\begin{equation*}
  \diffp{z}{u} \cdot \diffp{z}{v}
  = \left( \diffp{z}{x} \right)^2 - \left( \diffp{z}{y} \right)^2.
\end{equation*}

\oppgave[V2013, Oppgave 1] Fra fysiske lover kan en se at om $K$ er et homogent
legeme i $\R^3$, så må temperaturen $T = T(x,y,z)$ i $K$ være en løsning til
\emph{varmelikningen}
%
\begin{equation*}
  k \left( \diff[2]{T}{x} + \diff[2]{T}{y} + \diff[2]{T}{z} \right)
  = \diffp{T}{t}
\end{equation*}
%
der $(x,y,z)$ er posisjonen i legemet, $t$ er tiden, og $k$ er en
materialkonstant. \medskip

Vis at $T(x,y,z,t) = 2x - y + z$ er en løsning til varmelikningen.

\oppgave[V2012, Oppgave 4] Vi sier at $f(x,y)$ er harmonisk hvis $f$ er to
ganger kontinuerlig deriverbar og
%
\begin{equation*}
  \diffp[2]{f}{x} + \diffp[2]{f}{y} = 0.
\end{equation*}
%
Vis at hvis $g$ er harmonisk så er også $f(x,y) = g(x^2 - y^2, 2xy)$ harmonisk.

% \oppgave[V2013, Oppgave 4] Vis sier at en funksjonen er $C^2$ om alle
% andreordens partiellderiverte er kontinuerlige. Vis at om $f$ er en $C^2$
% funksjon, så er $\curl \nabla f = \mathbf{0}$. Si klart fra hvordan du bruker at
% funksjonen er $C^2$. 
