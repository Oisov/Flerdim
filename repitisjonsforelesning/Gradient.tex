\begin{frame}
  \begin{tikzpicture}
    \begin{axis}[ xbar=0pt, /pgf/bar shift=0pt, legend style={ legend columns=4,
        at={(xticklabel cs:0.5)}, anchor=north, draw=none }, ytick={0,...,15},
      ytick style={draw=none},% <- added
      axis y line*=none, axis x line*=bottom, tick label
      style={font=\footnotesize}, legend style={font=\footnotesize}, label
      style={font=\footnotesize}, xtick style={draw=none},% <- added
      xtick={0,1,...,12}, width=.9\textwidth, bar width=3mm, y dir = reverse,
      xmin=0, xmax=13, area legend,
      y=5mm, enlarge y limits={abs=0.625},
      style={text=black}, every axis plot/.append style={fill},
      nodes near coords, nodes near coords,
      yticklabels={%
        {\topicref{Grenseverdier}},
        {\topicref{Gradient}},
        {\topicref{Epsilon-delta}},
        {\topicref{Kjerneregelen}},
        {\topicref{Taylor-approksimasjon}},
        {\topicref{Tangenter}},
        {\topicref{Kritiske-punkter}},
        {\topicref{Optimering}},
        {\topicref{Linjeintegral}},
        {\topicref{Konservative-vektorfelt}},
        {\topicref{Dobbelintegraler}},
        {\topicref{Integrasjonsrekkefolge}},
        {\topicref{Trippelintegral}},
        {\topicref{Greens-teorem}},
        {\topicref{Divergensteoremet}},
        {\topicref{Stokes-teorem}}}]
      \addplot[fill=gray] coordinates {(8,0)};
      \addplot[fill=dgreen] coordinates {(4,1)};
      \addplot[fill=black] coordinates {(1,2)};
      \addplot[fill=black] coordinates {(7,3)};
      \addplot[fill=black] coordinates {(1,4)};
      \addplot[fill=black] coordinates {(7,5)};
      \addplot[fill=black] coordinates {(11,6)};
      \addplot[fill=black] coordinates {(4,7)};
      \addplot[fill=black] coordinates {(3,8)};
      \addplot[fill=black] coordinates {(8,9)};
      \addplot[fill=black] coordinates {(7,10)};
      \addplot[fill=black] coordinates {(3,11)};
      \addplot[fill=black] coordinates {(4,12)};
      \addplot[fill=black] coordinates {(4,13)};
      \addplot[fill=black] coordinates {(8,14)};
      \addplot[fill=black] coordinates {(6,15)};
    \end{axis}
  \end{tikzpicture}
\end{frame}

\begin{frame}
  \subsection{Gradient}\label{subsec:Gradient}
  \frametitle{Gradient}
  Gradienten til et skalarfelt er et vektorfelt som i hvert
  punkt peker i retningen til den største økningen i skalarfeltet.
  %
  \begin{align*}
       \nabla f (x,y)
    &= \Bigl( \diffp{f}{x}, \ \diffp{f}{y}\Bigl) \\
       \nabla f (\rho, \theta, z)
    &= \Bigl( \diffp{f}{\rho},\
             \frac{1}{\rho}\diffp{f}{\varphi},
             \diffp{f}{z}
       \Bigl) \\
       \nabla f (r, \theta, \phi)
    &= \Bigl( \diffp{f}{r},\
             \frac{1}{r}\diffp{f}{\theta},\
             \frac{1}{r \sin\theta}\diff{f}{\varphi}
       \Bigl)
  \end{align*}
\end{frame}

\begin{frame}
  \begin{oppgave}{V2016, Oppgave 1}
Anta at et fjell har form som en elliptisk paraboloide $z = c - ax^2 - by^2$,
der $a$, $b$ og $c$ er positive konstanter, $x$ og $y$ er øst–vest og nord–sør
koordinater på kartet, mens $z$ er høyden over havet.
%
\begin{enumerate}
  \item I hvilken retning stiger høyden mest i punktet $(1, 1)$?
  \item Hvis en klinkekule slippes i $(1, 1, c - a - b)$, i hvilken retning vil den begynne
    å trille?
\end{enumerate}
\end{oppgave}
\begin{enumerate}
  \item
  \visible<2-4>{Høyden til $z$ stiger mest i retningen til gradienten $\nabla z(1,1)$ i
  $(1,1)$.}
\visible<3-4>{
  \begin{equation*}
    \nabla z(1,1)
    = (-2ax, -2by)\mid_{(1,1)}
    \ = 2 (-a, -b)
  \end{equation*}
  %
  Slik at høyden stiger mest i retning $(-a,-b)$.
}
\visible<4>{\item Klinkekulen vil trille i den retningen høyden avtar raskest,
altså i retning $-\nabla z(1,1)$ med andre ord $(a,b)$.}
\end{enumerate}

\end{frame}
